
%*******************************************************
% Prefazione 
%*******************************************************
\chapter*{Prefazione}
\markboth{\spacedlowsmallcaps{Prefazione}}{\spacedlowsmallcaps{Prefazione}} 
\mtcaddchapter[\numberline{}\tocEntry{Prefazione}]
%\addtocontents{toc}{\protect\vspace{\beforebibskip}} % to have the bib a bit from the rest in the toc
%\addcontentsline{toc}{chapter}{\numberline{}\tocEntry{Prefazione}}

Queste note nascono come rielaborazione del materiale da me presentato ad alcune
conferenze che ho avuto occasione di svolgere alcuni anni passati per gli
Studenti delle classi quinte del Liceo Scientifico Statale G.~Marinelli di Udine
(Italia).  L'obiettivo degli interventi e di queste note \`e quello di cercare
di offrire una sintesi organica di alcuni tra gli aspetti principali legati allo
studio dei modelli fisici di evoluzione stellare.

Ho sempre ritenuto che l'approccio \emph{descrittivo} spesso adottato nei testi
e nei corsi di scienze delle scuole superiori per presentare l'argomento
dell'evoluzione stellare sia inutilmente limitativo.  Si tratta, infatti, di un
approccio perlopi\`u \emph{nozionistico},  incentrato sull'idea di
\emph{informare} lo Studente, anche dettagliatamente, delle singole tappe di
sviluppo di una stella, senza per\`o entrare pi\`u propriamente nel merito del
perch\'e si ritiene che la vita di una stella si svolga in tal modo e
soprattutto del come gli scienziati facciano a saperlo!

Studenti  dell'ultimo anno del Liceo hanno infatti abuto modo di maturare un
background di conoscenze di fisica sufficiente che permetta loro di afferrare, a
grandi linee almeno, gli aspetti fisici chiave che oggi si \`e propensi a
ritenere guidino la dinamica, la struttura e l'evoluzione delle stelle.

Non vi sarebbe ovviamente nulla di male in un simile approccio descrittivo se la
conoscenza attuale derivasse dall'osservazione diretta. In tal caso, non si
farebbe altro che limitarsi a presentare i risultati delle osservazioni,
lasciando l'eventuale interpretazione di queste osservazioni a studi successivi.
Si potrebbe essere portati a ritenere che buona parte delle conoscenze oggi
acquisite in merito all'evoluzione stellare siano il frutto di osservazioni
dirette. Non \`e cos\`{\i}{}. L'intera vita di una stella si svolge su scale di
tempo dell'ordine di parecchi milioni di anni. Evidentemente, simili durate di
tempo sono troppo lunghe per permettere di monitorare, in modo continuativo,
l'evolversi di una stessa stella in tutte le fasi del suo sviluppo. Invece,
quanto possiamo affermare di sapere oggi sull'evoluzione stellare discende in
larga misura dall'elaborazione di modelli teorici. Certamente, questi modelli
devono poi risultare compatibili con i dati oggi in possesso sugli oggetti
stellari conosciuti.  Il cielo ci offre costantemente esempi di oggetti stellari
che corrispondono alle varie tappe del processo evolutivo di una stella: stelle
in sequenza principale, stelle in formazione, nane bianche, giganti rosse,
supergiganti.  Quello che il cielo non pu\`o offrici in un unica stella \`e la
possibilit'`a di inseguirne il percorso evolutivo in tutte le sue singole tappe.
\`E stato fatto altrove questo esempio che coglie a mio avviso il punto chiave
della questione: osservare le stelle \`e un po' come osservare le persone allo
stadio; vi troverete persone di tutte le et\`a e potrete descriverle
dettagliatamente, ma questo di per s\'e non significa saper ordinare la tappe in
un percorso evolutivo. Questo tipo di ordine sequenziale richiede elaborazione
delle informazioni sperimentali e di modelli teorici da confrontare coi dati
osservativi.

Il testo \`e stato inizialmente pensato come materiale integrativo e di
approfondimento rispetto ai programmi scolastici tradizionalmente svolti nelle
classi quinte di un Liceo Scientifico.  Questa \`e anche la ragione per cui si
\`e scelto di mantenere l'esposizione a un livello piuttosto elementare,
omettendo in larga misura i dettagli tecnici e il formalismo matematico
richiesti, invece, in trattazioni di livello pi\`u avanzato.

%Per quanto riguarda il modo con cui i diversi argomenti vengono affrontati, \`e
%forse opportuno menzionare che si \`e fatto ricorso, piuttosto  frequentemente
%nell'esposizione, a nozioni di fisica elementare.  In un'epoca in cui
%l'interdisciplinariet\`a viene elevata a uno dei motivi centrali delle attuali
%metodologie didattiche, non pare soddisfacente avviare una presentazione
%dell'evoluzione stellare senza neppure fare menzione dei meccanismi fisici che
%ne sono alla base.

Vi \`e almeno una seconda ragione per prediligere un simile approccio
espositivo.  In questo modo, infatti, il testo non viene ridotto a essere una
presentazione informativa e perlopi\`u nozionistica, ma lascia intravvedere,
seppure solamente a grande linee, il ragionamento che  ``sta dietro'' allo
studio dell'evoluzione stellare.  


I risultati di fisica a cui si fa riferimento nel testo dovrebbero essere
padroneggiati con sicurezza da uno studente del quinto anno di Liceo
Scientifico, con l'eccezione, forse, del teorema del viriale.  Per ogni
evenienza, alcune note di supporto sono comunque raccolte in appendice.

L'approccio adottato \`e strettamente teorico.  Questo significa che i
riferimenti a oggetti astronomici osservati sono perlopi\`u assenti e, laddove
presenti, rimangono comunque  piuttosto marginali. Si rimanda il Lettore che
volesse approfondire questi aspetti ai pi\`u appropriati testi del settore.
Riferimenti completi ad alcuni di questi testi si possono reperire in
Bibliografia.  In particolare, si segnala l'ottimo volume di Robert
Burnham.

Per quanto riguarda l'organizzazione del materiale, il testo si articola in
cinque capitoli e quattro appendici.

Il \chaptername~\ref{nascita} presenta una panoramica sui processi di formazione
stellare. Il teorema del viriale per nubi autogravitanti \`e introdotto, senza
dimostrazione (\seename~\appendixname~\ref{app:viriale}), sin dall'inizio. Buona
parte dei risultati che saranno ottenuti nel seguito fanno riferimento a questo
teorema.  Buona parte della trattazione sin qui svolta si ispira all'eccellente
esposizione di Kittel.  Il materiale proposto contiene alcune
lacune.  Una delle omissioni maggiori riguarda il ruolo svolto dalla turbolenza
e dai campi magnetici nel collasso gravitazionale.  Si \`e preferito non
sviluppare per esteso questa parte, dati gli obiettivi modesti che questa
presentazione si propone.  Il lettore pu\`o integrare le informazioni al
riguardo consultando, ad esempio, un testo di livello intermedio
come.

Il \chaptername~\ref{stabilita} si rivolge alle stella in sequenza principale.
Non vengono introdotte le equazioni differenziali di struttura stellare.
Piuttosto, si \`e preferito dare risalto alla nozione di equilibrio,
distinguendo tra equilibrio idrostatico ed equilibrio termico.  \`E
opportuno prestare attenzione a questi termini: rispetto al significato
specifico che assumono in altri contesti (in termodinamica, ad esempio), vengono
adoperati qui con un'accezione piuttosto particolare.  Nella stessa accezione
saranno impiegati anche nei capitoli successivi.  Per il processo di fusione dei
nuclei di idrogeno, non si entra nei dettagli della catena protone--protone e
del ciclo di Bethe. La distinzione tra i caratteri che esibiscono le stelle in
cui \`e attivo prevalentemente l'uno o l'altro processo \`e presentata piuttosto
estesamente, seppure a un livello qualitativo.  Conclude il capitolo una
digressione sul caso, in parte ancora misterioso, di $\eta$ Carinae.

Il \chaptername~\ref{esodo} tratta l'esodo dalla sequenza principale.  Si \`e
preferito postporre la discussione sulla pressione di degenerazione degli
elettroni fino alla \sectionname~\ref{degenere}.  La formula per il calcolo
della pressione, derivabile nell'ambito della statistica quantistica di
Fermi--Dirac, non viene dimostrata. Piuttosto, buona parte della trattazione \`e
spesa per mettere in evidenza le peculiari propriet\`a della pressione di
degenerazione. \`E importante che queste propriet\`a vengano assimilate a fondo
dal Lettore: la pressione di degenerazione gioca infatti un ruolo centrale in
tutta la dinamica successiva dell'oggetto stellare.

Per quanto riguarda il fenomeno supernova, la dinamica dell'evento non viene
sviluppata nel dettaglio. \`E parso importante focalizzate l'attenzione sulle
differenze tra eventi di tipo Ia e di tipo Ib, questo soprattutto visto
l'impiego degli eventi Ia come candele standard per la misura delle distanze
extragalattiche.  L'esposizione \`e arricchita da alcuni eventi storici di
supernovae galattiche, e viene discusso l'evento 1989A.

\`E parso utile predisporre un quadro di riepilogo delle propriet\`a
caratteristiche delle nane bianche (\sectionname~\ref{nane bianche}). La
trattazione, ispirata ancora una volta all'eccellente volume di Robert
Burnham~\Cite{burnham}, offre al Lettore una buona occasione per abbandonare il
terreno della speculazione teorica e confrontarsi per un attimo con quanto
realmente ci \`e dato conoscere dall'osservazione del cielo. Il Lettore potr\`a
cos\`{\i} familiarizzare con gli ordini di grandezza in gioco.

Apre il \chaptername~\ref{chandra} una modesta introduzione dedicata alla figura
di Subramohion Chandrasekhar. Le vicende che hanno accompagnato la scoperta da
parte di Chandrasekhar della massa limite che oggi porta il suo nome meritano di
essere almeno menzionate.  La prima parte del capitolo \`e riservata agli
aspetti teorici del limite di Chandrasekhar. Pur non entrando nei dettagli
tecnici dell'argomento, si \`e cercato di enfatizzare il carattere
intrinsecamente relativistico del limite di Chandrasekhar: un gas degenere
relativistico, a differenza di un gas degenere non relativistico,  non \`e in
grado di fornire ''abbastanza`` pressione per garantire l'insorgere di una
condizione di stabilit\`a (idrostatica) nell'edificio stellare.  Per
l'eccellente qualit\`a della narrazione, la vicenda della scoperta sperimentale
della prima pulsar e le vicende che ne seguirono sono riportate, quasi parola
per parola, da~\Cite{encicl}.  La studio delle  propriet\`a della materia
neutronica nelle condizioni in cui viene a trovarsi nell'interno di una stella
di neutroni non sono neppure accennate. L'Autore non esita a dichiarare la sua
completa ignoranza al riguardo nel momento della stesura!  Il modello
Pacini--Glod e tratteggiato nei suoi aspetti fondamentali. Un'esposizione pi\`u
dettagliata pu\`o trovarsi, ad esempio, in un testo classico di radioastronomia
amatoriale come~\Cite{krauss}.

Le appendici si configurano come note integrative o di supporto ai contenuti
principali discussi nel testo.  L'\appendixname~\ref{app:cnero} tratta della
fisica del corpo nero. L'argomento si trova sviluppato in qusi ogni buon testo
di elettromagnetismo e fisica quantistica. Ci limiteremo qui a riportare alcuni
dei risultati pi\`u importanti a cui si \`e fatto riferimento nel testo,
ommettendo in molti casi le relative dimostrazioni.
L'\appendixname~\ref{app:neutrini} \`e una sommaria presentazione circa alcuni
recenti risultai usll'osservazione dei neutrini solari.
L'\appendixname~\ref{app:viriale} contiene una dimostrazione del teorema del
viriale, basata sull'uso delle leggi di Newton.  Infine,
l'\appendixname~\ref{app:buchi neri} tratta della fisica quantistica dei buchi
neri.

Sar\`o grato a chiunque voglia segnalarmi omissioni e/o errori sicuramente
presenti nel testo.  Qualunque consiglio o suggerimento sar\`a sempre pi\`u che
gradito.

\bigskip
 
\noindent\textsw{\myLocation, \myTime}

\begin{flushright}
    %\begin{tabular}{c}
        \Large{\calligra\myName} 
    %\end{tabular}
\end{flushright}

