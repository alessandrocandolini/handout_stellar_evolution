
% ********************************************************************
% General commands
% ******************************************************************** 
% un ambiente ad hoc per le citazioni
\newenvironment{citazione}%
  {\begin{quotation}\small\ignorespaces}%
  {\end{quotation}}

% [...] ;-)
\newcommand{\omissis}{[\dots\negthinspace]}

% indica a LaTeX la cartella dove sono riposte le immagini
\graphicspath{{./},{./Asymptote/}, {./Images/}, {./Metapost/}}

% eccezioni all'algoritmo di sillabazione
\hyphenation{Fortran ma-cro-istru-zio-ne nitro-idrossil-amminico}


% un ambiente ad hoc per gli approfondimenti
\newenvironment{approfondimento}%
  {\begin{quotation}\small\ignorespaces}%
  {\end{quotation}}

% i.e.
\newcommand{\ie}{i.\,e.}
\newcommand{\Ie}{I.\,e.}
\newcommand{\eg}{e.\,g.}
\newcommand{\Eg}{E.\,g.} 

% th
\newcommand{\ordth}{\textsuperscript{th}}


% ********************************************************************
% hyperref
% ******************************************************************** 
\hypersetup{%
    colorlinks=true, linktocpage=true, pdfstartpage=1, pdfstartview=FitV,%
    breaklinks=true, pdfpagemode=UseNone, pageanchor=true, pdfpagemode=UseOutlines,%
    plainpages=false, bookmarksnumbered, bookmarksopen=true, bookmarksopenlevel=1,%
    hypertexnames=true, pdfhighlight=/O,%
    urlcolor=webbrown, linkcolor=RoyalBlue, citecolor=RoyalBlue, pagecolor=RoyalBlue,%
% uncomment the following line if you want to have black links (e.g., for printing)
% urlcolor=Black, linkcolor=Black, citecolor=Black, pagecolor=Black,%
    pdftitle={\myTitle},%
    pdfauthor={\textcopyright\ \myName},%
    pdfsubject={},%
    pdfkeywords={},%
    pdfcreator={pdfLaTeX},%
    pdfproducer={LaTeX con hyperref e ClassicThesis}%
}

\hypersetup{citecolor=webgreen}
\hypersetup{hyperfootnotes=false,pdfpagelabels}

\newcommand{\mail}[1]{\href{mailto:#1}{\texttt{#1}}}


% *****************************************************************************
% Matematica 
% *****************************************************************************

% AMSmath packages 
\usepackage{amssymb}
%\usepackage{amsmath}

% comandi per gli insiemi numerici (serve il pacchetto amssymb)
\newcommand{\numberset}{\mathbb} 
\newcommand{\N}{\numberset{N}} 
\newcommand{\Z}{\numberset{Z}} 
\newcommand{\Q}{\numberset{Q}} 
\newcommand{\R}{\numberset{R}} 
%\newcommand{\C}{\numberset{C}} 

% Dirac notation  (serve il pacchetto braket)
\usepackage{braket} 
\newcommand{\modul}[1]{\mathinner{\vert#1\vert}} 
\newcommand{\Modul}[1]{\left\vert#1\right\vert} 
\newcommand{\norm}[1]{\mathinner{\Vert#1\Vert}}
\newcommand{\Norm}[1]{\left\Vert#1\right\Vert}
\newcommand{\conj}[1]{#1^{*}} 
\newcommand{\adj}[1]{#1^{\dagger}}

% un ambiente per i sistemi
\newenvironment{sistema}%
  {\left\lbrace\begin{array}{@{}l@{}}}%
  {\end{array}\right.}

% pacchetto cool 
\usepackage{cool}
\makeatletter
\Style{%
ArcTrig=arc,
IntegrateDifferentialDSymb={{\operator@font d}},
DSymb={{\operator@font d}},
DDisplayFunc=inset,DShorten=true}
\makeatother

% vectors (boldface style, primes are printed also in boldface)
%\usepackage[veceuler]{utvec}
\makeatletter
\def\ud{%
  \@ifnextchar[{\p@ud}{\np@ud}
}
\def\p@ud[#1]#2{%
\mathop{\kern\z@\operator@font d}%
  \csname nolimits@\endcsname^{#1}\!#2}

\def\np@ud#1{%
\mathop{\kern\z@\operator@font d}%
  \csname nolimits@\endcsname\!#1}

\def\udiff{%
  \@ifnextchar[{\p@udiff}{\np@udiff}
}
\def\p@udiff[#1]#2{%
\mathinner{\ud[#1]{#2}}}
\def\np@udiff#1{%
\mathinner{\ud{#1}}}

\def\uexp{
\mathop{\kern\z@\operator@font e}\nolimits}
%\newcommand{\udiffo}[2][\empty]{\mathinner{%
%  \mathop{\operator@font d\vphantom{exp}}\nolimits^{#1}\!#2}}
\newcommand{\deriv}[3][\empty]{%
    \frac{\ud[#1]{#2}}{\ud[\vphantom{#1}]{#3^{#1}}}}
 \makeatother


\usepackage[framemethod=tikz]{mdframed}

\MakeRobust\vec



% *****************************************************************************
% Teoremi e definizioni 
% *****************************************************************************

\usepackage{amsthm}                       % Serve il pacchetto amsthm.

\makeatletter
\newtheoremstyle{classicdef}%             % Stile tipografico dei teoremi
{11pt}%                                   % Spazio che precede l'enunciato
{11pt}%                                   % Spazio che segue l'enunciato
{}%                                       % Stile del font dell'enunciato
{}%                                       % Rientro (se vuoto, nessun rientro;
%                                         % \parindent = rientro dei capoversi)
{\scshape}%                               % Font dell'intestazione
{:}%                                      % Punteggiatura dopo l'intestazione
{.5em}%                                   % Spazio che segue l'intestazione:
%                                         % " " = normale spazio inter-parola;
%                                         % \newline = a capo
{}%                                       % Specifica intestazione enunciato
\makeatother

%\theoremstyle{definition}
\theoremstyle{classicdef}
\newtheorem{theorem}{Teorema}[chapter]
\newtheorem{lemma}{Lemma}[chapter]
\newtheorem{definition}{Definizione}[chapter]
\newtheorem*{homework}{Homework}
\theoremstyle{remark}
\newtheorem*{remark}{Nota}
\renewcommand{\qedsymbol}{\rule{.5em}{.5em}}

% *****************************************************************************
% Exercise environment 
% *****************************************************************************


\pgfdeclarehorizontalshading{exercisebackground}{100bp}
{color(0bp)=(blue!20);
color(100bp)=(orange!20)}

\pgfdeclarehorizontalshading{exercisetitle}{100bp}
{color(0bp)=(red!70);
color(100bp)=(black!5)}

  \mdfdefinestyle{theoremstyle}{%
     linecolor=red,linewidth=2pt,%
     frametitlerule=true,%
 %    apptotikzsetting={\tikzset{mdfframetitlebackground/.append style={%
  %                       shade,left color=white, right color=blue!20}}},
   %  frametitlerulecolor=green!60,
   %  frametitlerulewidth=1pt,
  %   innertopmargin=\topskip,
  frametitlefont=\scshape,%
   }
%\mdtheorem[style=exercisestyle]{Exercise}{Esercizio}
%\mdtheorem[style=theoremstyle]{Exercise}{Esercizio}

\providecommand{\problemname}{Problem}
\newcounter{problem}[chapter]
\setcounter{problem}{0}
\makeatletter
 \renewcommand{\theproblem}{%
\ifnum \c@chapter>\z@ \thechapter.\fi \@arabic\c@problem
}

\def\Exercise{%
  \addtocounter{problem}{1}
  \begin{mdframed}[backgroundcolor=lightgray!40, roundcorner=5pt, 
skipabove=12pt,skipbelow=12pt, 
%tikzsetting={shading=exercisebackground}
]
  \@ifnextchar [%
      {\@myprobt}%
      {\@myprob}%
}

\def\@myprobt[#1]{%
%\trivlist\item[\hskip\labelsep{\bfseries\problemname~\theproblem~(#1).}]}
\textcolor{orange}{\textsc\problemname~\theproblem (#1).}
}

\def\@myprob{%
%\trivlist\item[\hskip\labelsep{\bfseries\problemname\ \theproblem.}]
\textcolor{orange}{\textsc\problemname~\theproblem.}
}

\def\endExercise{\end{mdframed}}

\makeatother



% ********************************************************************
% Frontespizio
% ******************************************************************** 
\usepackage[suftesi,noadvisor]{frontespizio}         % frontespizo



\newtheorem*{Solution}{Soluzione}%todo

%\newtheorem{Exercise}{Esercizio}

% *****************************************************************************
% biblatex 
% *****************************************************************************

\renewcommand{\nameyeardelim}{, }

\defbibheading{bibliography}{%
\cleardoublepage
\manualmark
\phantomsection
%\addcontentsline{toc}{chapter}{\numberline{}\tocEntry{\bibname}}
\mtcaddchapter[\numberline{}\tocEntry{\bibname}]
\myChapter*{\bibname\markboth{\spacedlowsmallcaps{\bibname}}
{\spacedlowsmallcaps{\bibname}}}}     

% *****************************************************************************
% caption
% *****************************************************************************
\usepackage{caption}                      % Fancy captions and more.
\captionsetup{format=hang,font=small}
\captionsetup[table]{skip=\medskipamount} 


% *****************************************************************************
% makeidx, multicol
% *****************************************************************************
\let\orgtheindex\theindex
\let\orgendtheindex\endtheindex
\def\theindex{%
	\def\twocolumn{\begin{multicols}{2}}%
	\def\onecolumn{}%
	\clearpage
	\orgtheindex
}
\def\endtheindex{%
	\end{multicols}%
	\orgendtheindex
}

\makeindex

% *****************************************************************************
% breqn
% *****************************************************************************
\usepackage{mathtools}                    % Add support for cramped,
					  % mathlap,etc.
					  
\usepackage{siunitx}                      % Add support to print SI units
\usepackage[euler]{flexisym}              % Add support to Euler font
\usepackage{breqn}                        % Breqn

\makeatletter
   \def\eqnumsize{\normalfont \Tf@font}      % Add support to Minion Pro
\makeatother

\setkeys{breqn}{labelprefix={eq:}}


% *****************************************************************************
% Migliorare il kerning dell'apostrofo coi font MinionPro
% *****************************************************************************
\makeatletter 
\catcode`\'=12 
\def\qu@te{'} 
\catcode`'=\active 
\begingroup 
\obeylines\obeyspaces% 
\gdef\@resetactivechars{% 
\def^^M{\@activechar@info{EOL}\space}% 
\def {\@activechar@info{space}\space}% 
}% 
\endgroup 
\providecommand{\texorpdfstring}{\@firstoftwo} 
\protected\def'{\texorpdfstring{\active@quote}{\qu@te}} 
\def\active@quote{\relax 
  \ifmmode 
    \expandafter^\expandafter\bgroup\expandafter\prim@s 
  \else 
    \expandafter\futurelet\expandafter\@let@token\expandafter\qu@t@ 
  \fi} 
\def\qu@t@{% 
  \ifx'\@let@token 
    \qu@te\qu@te\expandafter\@gobble 
  \else 
    {}\qu@te{}\penalty\@M\hskip\expandafter\z@skip 
  \fi} 
\scantokens\expandafter{% 
  \expandafter\def\expandafter\pr@m@s\expandafter{\pr@m@s}} 
\makeatother


%%%%%%%%%%%
% margini %
% %%%%%%%%%%
% Suggestions from ClassicThesis
% Palatino 	10pt: 288--312pt | 609--657pt
% Palatino 	11pt: 312--336pt | 657--705pt
% Palatino 	12pt: 360--384pt | 768pt

\areaset[current]{336pt}{750pt}
\setlength{\marginparwidth}{7em}
\setlength{\marginparsep}{2em}%


%%%%%%%%%
% altro %
%%%%%%%%%

\hypersetup{citecolor=webgreen}
\hypersetup{pdfstartpage=1}

\newcommand{\cinterval}[2]{\mathinner{\left[#1,#2\right]}}
\newcommand{\ointerval}[2]{\mathinner{]#1,#2[}}

%\usepackage{mdframed}
\usepackage{nicefrac}
\usepackage{calligra}

\begin{asydef}
defaultpen(fontsize(10pt));
texpreamble("\usepackage[noopticals,onlytext]{MinionPro}");
texpreamble("\usepackage[small]{eulervm}");
texpreamble("\usepackage{nicefrac}");
texpreamble("\usepackage{cool}");
\end{asydef}

\makeatletter
\newcommand{\COOL@notation@ExpBaseESymb}{exp}% 'ln', 'log'
\newcommand{\COOL@notation@ExpShowBase}{at will}% 'at will', 'always'
\renewcommand{\Exp}[2][\E]
{%
\ifthenelse{ \equal{\COOL@notation@ExpShowBase}{at will} }%
{%
\ifthenelse{ \equal{#1}{\E} }%
{%
\ifthenelse{ \equal{\COOL@notation@ExpBaseESymb}{exp} }%
{%
\exp \COOL@decide@paren{Exp}{#2}%
}%
{%
\ifthenelse{ \equal{\COOL@notation@ExpBaseESymb}{exp} }%
{%
\exp \COOL@decide@paren{Exp}{#2}%
}%
{%
\PackageError{cool}{Invalid Option Sent}%
{ExpBaseESymb can only be `ln' or `log'}%
}%
}%
}%
{%
\ifthenelse{ \equal{#1}{10} \AND
\NOT \equal{\COOL@notation@ExpBaseESymb}{exp}  }%
{%
\exp \COOL@decide@paren{Exp}{#2}%
}%
{%
\exp_{#1} \COOL@decide@paren{Exp}{#2}%
}%
}%
}%
{%
\ifthenelse{ \equal{\COOL@notation@ExpShowBase}{always} }%
{%
\exp_{#1}\COOL@decide@paren{Exp}{#2}%
}%
{%
\PackageError{cool}{Invalid Option Sent}%
{LogShowBase can only be 'at will' or 'always'}%
}%
}%
}

\makeatother

\usepackage{esdiff}

\newcommand{\df}{}

\makeatletter
\def\df{\@ifnextchar[{\ES@diffform@i}{\ES@diffform@ii}}
\def\ES@diffform@i[#1]#2{\mathchoice{%
\ES@dop\ES@difint#2^{#1}}%
{\ES@taille{\ES@dop\ES@difint#2^{#1}}}%
{\scriptstyle{\ES@dop\ES@difint#2^{#1}}}%
{\scriptstyle{\ES@dop\ES@difint#2^{#1}}}}

\def\ES@diffform@ii#1{\mathchoice%
{\ES@dop\ES@difint#1}%
{\ES@taille{\ES@dop\ES@difint#1}}%
{\scriptstyle{\ES@dop\ES@difint#1}}%
{\scriptstyle{\ES@dop\ES@difint#1}}}

\makeatother
