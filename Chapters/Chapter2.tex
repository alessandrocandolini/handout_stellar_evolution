
%*******************************************************
% Chapter 2
%*******************************************************
\chapter{La sequenza principale \label{stabilita}}
\minitoc\mtcskip

\noindent Il processo di contrazione gravitazionale si arresta quando la temperatura nelle regioni centrali della stella è tale da consentire il verificarsi di reazioni di fusione nucleare.
La stella entra in sequenza principale e viene definita di \emph{et\`{a} zero}.
\section{Equilibrio}\label{equilibrio}
\subsection{Massa di Jeans. Nane brune}\label{Jeans}
\`E bene precisare sin da subito come non tutte le protostelle in fase di contrazione raggiungano temperature tali da consentire l'innescarsi al loro interno delle reazioni termonucleari; si stima che per masse inferiori a $0,07 M_{\odot}$, con $M_{\odot}$ massa del Sole, la protostella non raggiunga mai lo stadio di stella di sequenza principale.\footnote{La massa critica che una protostella in fase di contrazione deve  possedere per diventare, dopo un opportuno periodo di tempo,  una stella di et\`{a} zero è detta \emph{massa di Jeans}. Il valore qui riferito, pari a $0.07$ masse solari, è riportato in \Cite{rosino}, p. $750$. Altre fonti forniscono per la massa di Jeans il valore di $0.08 M_{\odot}$ \citep{hack}.} In questo caso infatti il collasso gravitazionale verr\`{a} arrestato (prima che si assista all'innesco delle reazioni di fusione)  da effetti di natura quantistica sui quali si discorr\`{a} più approfonditamente in seguito in riferimento alle nane bianche.
Per valori di massa $<0.07 M_{\odot}$ la protostella terminer\`{a} infine la propria vita nello stadio di nana bruna, irradiando lentamente energia nello spazio, raffreddandosi fino al sopraggiungere della morte termica.
\par
\`E interessante notare come anche un pianeta come Giove si possa far rientrare in quest'ultima categoria di oggetti stellari. Secondo una descrizione del pianeta gigante, risalente ancora agli anni tra il 1940 e il 1950 ma tutt'oggi rimasta largamente valida, la composizione dell'atmosfera gioviana si accorda perfettamente con una composizione simile a quella del Sole e di altre stelle. \footnote{Gli spettrometri infrarossi montati sulle sonde Voyager hanno permesso di stimare il rapporto fra abbondanza di elio ($He$) e idrogeno ($H_{2}$) pari a 0.11 $\pm$ 0.3, in ottimo accordo con il valore ottenuto per il Sole, che è pari a 0.12 \Cite{cav}. \`E doveroso segnalare che, in tempi più recenti, l'esplorazione  diretta dell'atmosfera gioviana ad opera di un modulo spaziale sganciato nel dicembre del 1995 dalla sonda Galileo ha messo in evidenza un sorprendentemente elevato contenuto di carbonio, azoto e ossigeno rispetto alla composizione dell'atmosfera del Sole. Pare plausibile, allo stadio attuale di riduzione dei dati, che impatti di comete e asteroidi col pianeta possano averne influenzato in maniera rilevante la composizione degli strati più esterni \Cite{conti}.}
A quanto pare, all'epoca della sua formazione Giove non raggiunse una massa sufficiente affinché il collasso gravitazionale del corpo potesse proseguire fino all'avvio, nel suo interno, delle reazioni di fusione. 
\subsection{Condizioni di equilibrio}\label{ce}
Nel caso in cui la massa in gioco sia superiore alla massa critica di Jeans, la temperatura della protostella diventa infine cos\`{\i}{} elevata da consentire il verificarsi dei processi di fusione nucleare.
Il questo caso la quantit\`{a} di energia persa per irraggiamento dalla stella viene compensata perfettamente da quella prodotta nel nocciolo centrale dalle reazioni nucleari. 
La variazione di energia totale $\Delta E$ del sistema è perciò nulla:
\begin{displaymath}
\Delta E = 0
\end{displaymath}
Conseguenza di ciò si assiste all'arresto del collasso gravitazionale e la stella entra in una fase di stabilit\`{a}.
Con stabilit\`{a} non si intende il raggiungimento di uno stato statico, ma piuttosto di una condizione \emph{stazionaria} di equilibrio dinamico in cui mediamente non avvengono cambiamenti significativi nelle caratteristiche e nella struttura complessiva della stella.
La fase di stabilit\`{a} risulta caratterizzata da \citep{hack}:
\begin{enumerate}
\item
\emph{Equilibrio termico}: tanta energia viene prodotta dalle reazioni di fusione nucleare tanta viene emessa dalla superficie. In altri termini l'energia prodotta dalle reazioni di fusione è sufficiente a mantenere costante la temperatura della stella.
\item
\emph{Equilibrio idrostatico}: in termini di forze, la forza di attrazione gravitazionale della stella su sé stessa è perfettamente controbilanciata dalla forza di pressione\footnote{Più propriamente ciò che bilancia la forza di gravit\`{a} è il \emph{gradiente} di pressione} che tende a far espandere la stella stessa.
\par
Si consideri a titolo di esempio un qualche elemento del corpo stellare avente massa $dm$ e volume infinitesimo $dV$. Le forze agenti su questo elemento sono, come schematicamente raffigurato in figura 1, la forza di attrazione gravitazionale $\vec{F_{G}}$ e la forza di pressione $\vec{F_{p}}$. Affinché vi sia equilibrio idrostatico la risultante delle forze agenti deve essere nulla, cioè $F_{G}=F_{p}$ indipendentemente da quale elemento di massa si scelga di considerare.
\footnote{Per il lettore che lo desiderasse, si riporta, a puro titolo di curiosit\`{a}, l'equazione dell'equilibrio idrostatico ricavabile imponendo che la forza di attrazione gravitazionale $F_{G}$ agente su un elemento di massa $dm$ per comodit\`{a} scelto a forma di parallelepipedo, posto ad una distanza $r$ dal centro della stella e avente volume infinitesimo $dV=dr \cdot dS$, eguagli il gradiente di pressione:
\begin{displaymath}\nonumber
\frac{dP(r)}{dr}=-G \frac{M(r) \rho(r)}{r^{2}}
\end{displaymath}
dove $M(r)$ è la massa della stella contenuta entro il raggio $r$, $\rho(r)$ la densit\`{a} alla distanza $r$, $P(r)$ la pressione alla distanza $r$ e $G$ la costante di gravitazione universale.
}
\end{enumerate}
%\begin{figure}[h]
%\begin{center}
%\includegraphics{equilibrio4}
%\end{center}
%\caption{Equilibrio idrostatico.}
%\end{figure}
La forza di pressione è data fondamentalmente dalla somma di due contributi, il primo dei quali dovuto alla pressione del plasma stellare: non diversamente da quanto avviene per un gas che esercita una pressione sulle pareti del recipiente che lo contiene, cos\`{\i}{} allo stesso modo anche il plasma stellare esercita una pressione in virtù del fatto che  le particelle che lo compongono hanno una certa energia cinetica.
Una seconda sorgente di pressione è data dalla \emph{pressione di radiazione}.% che, prodotta nel nocciolo centrale della stella, cerca di raggiungere la superficie esterna. La pressione di radiazione sar\`{a} approfondita più dettagliatamente al numero \ref{prad}.
\subsection{Pressione di radiazione}\label{prad}
Nel seguito chiameremo $P$, per brevit\`{a}, la pressione esercitata dal plasma. Si indicher\`{a} invece con $P_{rad}$ la pressione di radiazione.
\par
Si può dimostrare che il rapporto $P_{rad}/P$ è all'incirca direttamente proporzionale al quadrato della massa $M$ della stella: \footnote{La relazione è deducibile nell'ipotesi che per il plasma stellare possa ritenersi valida l'equazione di stato del gas perfetto, cioè nell'ipotesi che il plasma si \emph{comporti} (ovviamente non che realmente sia!) come un gas ideale.}
\begin{equation}\label{pressione_radiazione}
\frac{P_{rad}}{P} \propto M^{2}
\end{equation}
Detto a parole, maggiore è la massa $M$ della stella, (e dunque maggiore è $M^{2}$) qualitativamente più grande sar\`{a} il rapporto tra pressione di radiazione e pressione del gas. Questo è, in breve, il significato della relazione (\ref{pressione_radiazione}): via che si considerano stelle di massa sempre maggiore, il rapporto tra pressione di radiazione e pressione del plasma tende ad aumentare in misura sempre più significativa.
\par
Si è voluto insistere su questo aspetto, riportando anche la relazione matematica (\ref{pressione_radiazione}), perch\'{e} esso ha una interessante conseguenza, e cioè che il contributo della pressione di radiazione alla pressione complessiva diventa particolarmente importante quando si ha a che fare con stelle massicce. 
\par
Il fatto che la pressione di radiazione possa diventare dominante in stelle molto massicce è alla base di alcuni accreditati modelli sul meccanismo di formazione dei \emph{venti stellari}: pare plausibile che, per effetto dell'intensa pressione di radiazione, stelle massicce subiscano l'espulsione nello spazio di ingenti quantit\`{a} di materiale stellare (alcune volte in maniera violenta, altre volte in modo lento e prolungato). L'emissione cessa una volta che la massa della stella si riduce ad un valore sufficientemente piccolo da rendere praticamente trascurabili, in base alla relazione (\ref{pressione_radiazione}), gli effetti della pressione di radiazione. \footnote{Il vento solare è invece di origine \emph{termica}: si sviluppa nella corona causa l'altissima temperatura che caratterizza questa regione del Sole. Come fatto notare ancora da E. N. Parker, l'elevata temperatura della corona (compresa tra uno e due milioni di Kelvin) fa  si che l'attrazione gravitazionale solare non sia sufficiente a mantenere confinato il gas coronale. Per una trattazione qualitativa più approfondita sui venti stellari si veda ad esempio \Cite{WE}.}
\par
La dipendenza lineare del rapporto $P_{rad}/P$ dal quadrato della massa della stella fissa anche la massa massima $M_{max}$ che una stella può avere, che risulta essere pari a:
\begin{displaymath}
M_{max}\sim 100 M_{\odot}
\end{displaymath}
\subsection{Comportamento ``autoregolante'' di una stella di sequenza principale}
Vi è ancora un aspetto degno di nota sulle caratteristiche di una stella in fase di stabilit\`{a}.
\par
Durante la permanenza della stella in sequenza principale, la stella costituisce un sistema \emph{autoregolante}. Essa possiede un sistema termostatico che le consente di mantenersi nelle condizioni di equilibrio termico e idrostatico prima descritte: qualora casualmente il tasso di produzione di energia nel nucleo aumentasse, si avrebbe una espansione dei gas esterni, con conseguente raffreddamento e rallentamento dei processi di fusione. 
Analogamente un decremento significativo nell'attivit\`{a} di nucleosintesi sarebbe accompagnato da una contrazione del corpo stellare sufficiente a garantire un aumento della temperatura nel nocciolo centrale e il ripristino delle condizioni di equilibrio.
\section{Vita di una stella in sequenza principale}\label{sp}
Le stelle di et\`{a} zero occupano una stretta banda del diagramma HR, nota con il nome di sequenza principale. La posizione precisa in cui la stella fa la sua comparsa in sequenza principale fu studiata dal giapponese Hayashi;
egli calcolò come le coordinate di ingresso in sequenza principale siano strettamente dipendenti dal valore della massa della stella.
\footnote{Forse non sorprender\`{a} il lettore sapere che il risultato fu ottenuto da Hayashi proprio facendo uso, sotto certe ipotesi, del teorema del viriale. Del resto questo teorema si presta ad essere applicato in astronomia ad una grandevariet\`{a} di contesti e situazioni differenti. Basti pensare che una delle prove a sostegno dell'ipotesi dell'esistenza nelle galassie della cosiddetta materia oscura proviene proprio dal risultare sperimentalmente non rispettato il teorema del viriale per alcuni di questi oggetti cosmici.
\par
Una discussione quantitativa del percorso evolutivo tracciato da Hayashi è reperibile in \Cite{collins}.  Il lavoro originale di Hayashi è reperibile in: C. Hayashi, \emph{Evolution of Protostars}, Ann. Rev. Astr. and Astrophys. 4, 1966, pp. 171-192.}
\par
Più in generale, \emph{l'intero tracciato evolutivo di una stella è strettamente connesso al valore della sua massa}.
\par
Prescindendo dai dettagli tecnici del lavoro di Hayashi, la figura 2 illustra schematicamente l'ingresso in sequenza principale di protostelle di massa diversa in un diagramma HR; le masse sono espresse come multipli della massa solare $M_{\odot}$ \Cite{collins}.
%\begin{figure}\label{Hayashi}
%\begin{center}
%\includegraphics[width=15cm]{track1}
%\end{center}
%\caption{Diagramma HR per la transizione protostella--stella di sequenza principale secondo il modello Hayashi \Cite{collins}.}
%\end{figure}
Il tempo di permanenza di una protostella nella fase contrattiva è anch'esso un parametro che, come si potrebbe intuitivamente essere portati a pensare, è diverso a seconda dei valori di massa in gioco. In genere protostelle di massa maggiore evolvono più rapidamente verso lo stato di stelle di et\`{a} zero rispetto a protostelle di massa inferiore \Cite{rosino}.
\subsection{Stelle giganti blu}\label{giganti blu}
Le stelle molto massicce, situate nel diagramma HR nell'alto della sequenza principale, raggiungono nel nocciolo (\emph{core}) centrale temperature sufficientemente elevate\footnote{Maggiori di 15 milioni di kelvin secondo la stima riportata in: \Cite{rosino} p. 758} da consentire prevalentemente il ciclo\footnote{Si parla di ciclo perch\'{e} in esso il nucleo di carbonio funge da catalizzatore della nucleosintesi dell'elio, trasformandosi dapprima in un nucleo di azoto 14 per poi riconvertirsi in carbonio 12.} di Bethe ($CNO$); essendo la temperatura elevata, per la legge dello spostamento di Wien, appariranno di colore blu-azzurro, e per questa ragione vengono designate come \emph{giganti blu}\footnote{cfr. ad es.: M. Crippa, M. Fiorani, \emph{Geografia generale}, Arnoldo Mondadori Scuola, Milano 2002, p. 33}; nel diagramma HR occupano la porzione in alto a sinistra della sequenza principale.
\par
Per mantenere una temperatura interna ottimale al mantenimento di una condizione di equilibrio, le giganti blu consumano il combustibile nucleare del nocciolo più velocemente rispetto a stelle meno massicce e conseguentemente il loro tempo di permanenza in sequenza principale risulta estremamente ridotto, nelle scale di tempo di vita di una stella.
\par
Un ulteriore elemento di diversit\`{a} che distingue stelle massicce da stelle di massa inferiore riguarda le modalit\`{a} di propagazione del calore nel nocciolo interno.
\footnote{La diffusione di energia all'interno di una stella avviene attraverso le usuali modalit\`{a} di trasmissione:
\begin{enumerate}
\item
\emph{convezione}, cioè sviluppo di moti ascensionali di materiale caldo e discendenti di materiale freddo.
\item
\emph{irraggiamento}, cioè attraverso emissione e successivo riassorbimento di fotoni. Viene di norma riferito con l'espressione \emph{trasferimento radiativo}.
\item
\emph{conduzione} dovuta alla collisione tra particelle.
\end{enumerate}
In genere, se confrontata per efficacia con gli altri due meccanismi di diffusione, la conduzione può essere trascurata.}
Nel caso, sin qui preso in esame, di stelle massicce, il trasporto di calore nel core avviene prevalentemente per convezione. Per contro, nelle zone esterne al core risulta più efficace il trasporto radiativo.
Non si ha rimescolamento del materiale del core con quello degli strati esterni della stella.
\subsection{Stelle nane rosse}\label{nane rosse}
Si consideri ora il caso di stelle meno massicce, diciamo aventi massa indicativamente inferiore a due masse solari $M<2M_{\odot}$ \Cite{collins}; nel diagramma HR occupano la porzione più bassa della sequenza principale e per questo vengono alle volte riferite come stelle di bassa sequenza principale (vedi figura 2).
%
% Qui c'è scritto figura 2
%
La massa di queste stelle non è sufficiente a garantire il raggiungimento nel nocciolo centrale di temperature tali da consentire il funzionamento del ciclo $CNO$; il processo di fusione prevalente è allora la catena protone-protone (p--p).
\par
L'energia liberata dalla catena p--p è \emph{esattamente} la stessa che nel ciclo $CNO$ attivo in stelle più massicce.
Ciò che cambia nei due casi è il modo in cui la velocit\`{a} con cui le reazioni avvengono dipende dalla temperatura.
Nel caso della catena p--p, la velocit\`{a} della reazione cresce con la quarta potenza della temperatura assoluta (cioè con $T^{4}$) mentre per il ciclo $CNO$ la dipendenza dalla temperatura è assai più vincolante, essendo la velocit\`{a} della reazione proporzionale alla 20$^{a}$ potenza della temperatura assoluta (cioè a $T^{20}$ !) \Cite{rosino}.
\par
Questo fatto ha una importante conseguenza.
\par
In genere, sia che si considerino stelle massicce o stelle di bassa sequenza principale, la temperatura diminuisce abbastanza rapidamente via via che ci si sposta dal centro verso l'esterno della stella.
La dipendenza della velocit\`{a} del ciclo $CNO$ da $T^{20}$ fa si che una relativamente piccola diminuzione della temperatura $T$ abbia come effetto una diminuzione alquanto cospicua dell'efficienza del ciclo $CNO$ allontanandosi dalle regioni centrali. Un esempio numerico potr\`{a} aiutare a chiarire meglio il concetto: 
è dato il caso in cui il ciclo $CNO$ sia attivo in una stella la cui temperatura al centro è stimata essere pari a 20 milioni di Kelvin. Appena fuori dal centro la temperatura scende a 18 milioni di Kelvin.
A questa temperatura l'efficienza del ciclo $CNO$ (che dipende da $T^{20}$) si riduce a 1/8 rispetto al centro e ad appena un centesimo qualora la temperatura diminuisca ulteriormente a 16 milioni di Kelvin.
\footnote{l'esempio è tratto da: \Cite{rosino}, p. 763. Non dovrebbe essere particolarmente difficile ricavare i risultati presentati da Rosino: il rapporto tra l'efficienza ad una temperatura diciamo $T_{1}$ e l'efficienza ad una seconda temperatura diciamo $T_{2}$ è semplicemente $(T_{1} / T_{2})^{20}$.
Nel caso della diminuzione tra 20 milioni di Kelvin e 18 milioni di Kelvin, il rapporto tra l'efficienza a 20 milioni e l'efficienza a 18 milioni è $\left( (20 \cdot 10^{6})/(18 \cdot 10^{6}) \right) ^{20}= (20/18)^{20} \sim 8.22$.}
Nelle stelle massicce la sintesi di energia interessa pertanto solamente una alquanto contenuta regione del corpo stellare in prossimit\`{a} del suo centro.
\par
Non accade lo stesso laddove, per stelle di massa inferiore, predomina la catena p--p. La dipendenza della velocit\`{a} della reazione da $T^{4}$ fa si che i processi di sintesi e produzione di energia interessino una regione assai più estesa attorno al centro.
\par

\smallskip

Le stelle di bassa sequenza principale, dato il valore relativamente basso della temperatura, appaiono di colore rosso e per questo sono dette alle volte \emph{nane rosse}.
\par
Il trasporto di calore nelle regioni centrali avviene prevalentemente per trasporto radiativo.
Per queste stelle le condizioni per l'instaurarsi di regimi convettivi hanno modo di verificarsi a livello delle regioni più esterne. Per il Sole si stima che la transizione da regime prevalentemente radiativo a regime prevalentemente convettivo avvenga ad una distanza dal centro stimata pari a circa $0.75R_{\odot}$, dove $R_{\odot}$ è la misura del raggio solare \Cite{collins}.
\par
L'assenza pressoché completa nel core di moti di origine convettiva ha rilevanti implicazioni nell'evoluzione futura della stella; essa concorre a determinare un accumulo dell'elio $^{4}He$, prodotto dalla combustione dell'idrogeno, nelle regioni centrali.
In questo stadio di evoluzione, l'elio non può venire a sua volta impiegato per ulteriori processi di nucleosintesi\footnote{Come si dir\`{a} anche nel seguito, le temperatura richieste affinché abbiano luogo reazioni di fusione dei nuclei di elio sono pari a circa 100 milioni di kelvin \Cite{kittel}.}. Reazioni che comportino ad esempio la sintesi di $_{5}Li$ a partire dall'elio 4 per successiva cattura di protoni non possono invece avere luogo perch\'{e} i nuclei aventi numero di massa eguale a 5 non sono stabili \Cite{rosino}, mentre le scorte di idrogeno disponibile per le reazioni di fusione diventano limitate in misura sempre maggiore.
Il tasso di produzione di energia rimane stabile al valore ottimale perch\'{e} la stella permanga nella propria condizione di equilibrio solo a condizione che la temperatura nel nocciolo venga adeguatamente aumentata. L'aumento delle temperature al centro si traduce in un sensibile aumento della luminosit\`{a} della superficie emettente più esterna della stella. Si stima che il Sole abbia aumentato di circa il 40\% la sua luminosit\`{a} dall'epoca del suo ingresso in sequenza principale \Cite{collins}.
\par
Per stelle massicce le correnti convettive del core assicurano invece un sistema di smaltimento dell'elio sintetizzato e un continuo apporto di nuovo idrogeno indispensabile al mantenimento delle reazioni di fusione.
In ciascun caso comunque non ha modo di verificarsi un qualche rimescolamento tra il materiale del core (che progressivamente va arricchendosi in contenuto di elio) e quello (ancora ricco di idrogeno) degli strati esterni.
La continua diminuzione della quantit\`{a} di idrogeno effettivamente disponibile  nel core centrale condurr\`{a} infine alla situazione in cui le reazioni di fusione fin qui descritte, divenute altamente improbabili per l'assenza di ``combustibile'', risulteranno insufficienti a garantire l'apporto energetico indispensabile al mantenimento delle condizioni di equilibrio.
Si va cos\`{\i}{} preparando la strada all'esodo della stella dalla sequenza principale.
\par

\smallskip

In condizioni ordinarie, una stella permane nello stadio stabile di sequenza principale per una durata di tempo, indicabile con $\tau$ (notazione non standard), legata alla massa $M$ della stella da una relazione di proporzionalit\`{a}:
\begin{equation}\label{timelife}
\tau \propto \frac{1}{M^{3.5}}
\end{equation}
%
%\begin{figure}[!h]\label{efficienza1}
%\begin{center}
%\includegraphics[width=15cm]{efficienza1}
%\\
%\emph{Figura 3.} Efficienza del ciclo $CNO$ e della catena p--p per diversi valori della temperatura del nocciolo centrale della stella.
%\end{center}
%\end{figure}
%\newpage
\subsection{Riepilogo delle informazioni in sintesi}
Si riporta di seguito un riepilogo in sintesi dei principali elementi di diversit\`{a} che contraddistinguono stelle di bassa sequenza principale e stelle di alta sequenza principale. 
%\begin{center}
%\begin{tabular}{p{3mm} c p{3mm} c p{3mm}}
%\multicolumn{5}{c}{Tabella 1} \\
%\hline
%& & & & \\
%& \emph{Giganti blu} & & \emph{Nane rosse}  & \\[3.00mm]
%& & & & \\
%& T$_{\textrm{core}}$ $>$ 15 milioni di K && T$_{\textrm{core}}$ $>$ 10 milioni di K & \\
%& ciclo $CNO$ $^{*}$         && catena p--p   &  \\
%& colore blu-azzurro         && colore rosso    & \\
%& stelle di alta sequenza principale && stelle di bassa sequenza principale &  \\
%& core prevalentemente convettivo         && core prevalentemente radiativo    & \\
%&&&& \\
%& \multicolumn{3}{l} {$^{*}$ \footnotesize In misura minore si verifica anche la catena p--p} & \\[2.00mm]
%\hline
%\end{tabular}
%\end{center}
%In condizioni ordinarie, una stella permane nello stadio stabile di sequenza principale per una durata di tempo, indicabile con $\tau$ (notazione non standard), legata alla massa $M$ della stella da una relazione di proporzionalit\`{a}:
%\begin{equation}\label{timelife}
%\tau \propto \frac{1}{M^{3.5}}
%\end{equation}
\section{Determinare l'et\`{a} degli ammassi stellari dal loro diagramma HR}
La relazione (\ref{timelife}) viene usata per determinare l'et\`{a} di un ammasso di stelle dalla conformazione della sequenza principale nel diagramma HR osservativo dell'ammasso stesso.
In un ammasso stellare di formazione recente infatti le stelle più massicce (si da il caso che siano anche le più luminose) di classe spettrale $OB$, non possono gi\`{a} aver abbandonato la sequenza principale e pertanto saranno presenti nel diagramma HR relativo all'ammasso; al contrario per ammassi più vecchi (i cosiddetti ammassi globulari) è gi\`{a} trascorso un periodo di tempo sufficiente perch\'{e} queste stelle abbiano ormai lasciato la sequenza principale.
Nel diagramma HR osservativo di un ammasso globulare in genere è quasi del tutto assente il ramo superiore della sequenza principale, difettando l'ammasso delle stelle di classe $OB$; il punto dal quale la sequenza principale è troncata viene detto \emph{main sequence turn-off}. Le poche stelle che compaiono in sequenza al di sopra del punto di turn-off vengono indicate come \emph{blue straggers}, e rappresentano una classe di stelle per le quali si è verificato un rimescolamento del materiale del core con quello, ricco di idrogeno, degli strati più esterni. Il diagramma HR per ammassi globulari presenta alcuni ulteriori elementi di diversit\`{a} che  andrebbero sottolineati; l'argomento, ovviamente molto più vasto, esula dagli obiettivi del presente documento, e pertanto non sar\`{a} ulteriormente discusso. Il lettore che lo desiderasse può consultare \Cite{rosino}.
\section{Associazionismo T--Tauri. Lo strano caso di $\eta$ Carinae}
Precisiamo che la fase di transizione tra protostella e stella non avviene certo in forma cos\`{\i}{} tranquilla come potrebbe sembrare.
Molte giovani stelle, prima di raggiungere una configurazione di relativa stabilit\`{a}, attraversano un periodo di variabilit\`{a} in cui la loro luminosit\`{a} fluttua irregolarmente con ampiezza di alcune magnitudini. \`E questo il caso delle stelle cosiddette di associazione T (associazioni di stelle prevalentemente rosse che hanno il loro prototipo nella stella T Tauri) o di variabili come RW Aurigae.
Non mancano casi di stelle in contrazione che hanno aumentato la loro luminosit\`{a} intrinseca di un fattore cento o anche mille, ritornando poi dopo mesi o anni al loro originario splendore. Ne sono tipici esempi V 1057 Cygni e FU Orionis; quest'ultima nel 1937 aumentò di 6 magnitudini in un periodo di 19 giorni \Cite{rosino}.
\par
Secondo taluni forse anche la famosa stella australe $\eta$ Carinae, troppo debole per essere catalogata come supernova e troppo luminosa per essere considerata una nova, potrebbe rientrare in questa classe di stelle variabili in via di formazione; stimata da Halley di quarta grandezza, $\eta$ Carinae raggiunse nell'aprile del 1843 una luminosit\`{a} visuale apparente di -0.8 magnitudini, diventando la stella più brillante del cielo dopo Sirio; in seguito si affievol\`{\i}{}, diventando invisibile a occhio nudo gi\`{a} nel 1868; agli inizi del 1900 la sua magnitudine visuale apparente fu stimata attorno all'ottava grandezza; nel 1941 aumentò nuovamente la propria luminosit\`{a} e nel 1953 fu stimata di 7$^{a}$ magnitudine \Cite{burn}.
La curva di luminosit\`a \'e riportata in figure. 
L'ipotesi che si tratti di una stella di formazione recente sembra confermata dal fatto che essa appartenga ad una nebulosa diffusa, la nebulosa NGC 3372, e costituisca una delle sorgenti infrarosse più intense tra quelle oggi conosciute. Risposte definitive in questo senso comunque non sono ad oggi ancora state formulate.
%\begin{figure}[h]
%\begin{center}
%\includegraphics[width=12cm, height=50mm]{carinae}
%\end{center}
%\caption{Curva di luce di $\eta$ Carinae \Cite{carinae}.}\label{carinAe}
%\end{figure}
%\newpage



