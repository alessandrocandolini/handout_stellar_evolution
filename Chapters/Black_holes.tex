
%*******************************************************
% Chapter 4
%*******************************************************

\myChapter{Buchi neri} \label{chap:buchi neri}
\minitoc\mtcskip

\noindent Questo capitolo si propone un duplice scopo. 
Il primo  obiettivo \`e 
presentare alcuni concetti di base relativi alla fisica (classica) dei buchi
neri, che verranno illustrati  esplicitamente nel caso particolare della geometria di
Schwarzchild e commentando alcuni risultati pi\`u generali  che possono essere
derivati dallo studio della struttura causale globale.
Il secondo, \`e esporre la connessione che questi argomenti di fisica teorica
hanno con l'astrofisica. L'astrofisica offre infatti un contesto tipico dove
incontrare buchi neri, in particolare:
\begin{itemize}
   \item Buchi neri stellari (masse intorno a qualche volta la massa solare);
   \item Buchi neri supermassicci al centro di certe galassie;
   \item Buchi neri primordiali.
\end{itemize}
In particolare, entreremo pi\`u nel dettaglio sui buchi neri stellari (prodotti
dal collasso gravitazionale di stelle di massa superiore al limite di ).
I buchi neri supermassicci al centrod ella galassia sono rilevanti per spiegare
certi processi in astrofisica, ne accenneremo brevemente. 
Illustreremo anche alcune prove (indirette) molto forti a favore dell'esistenza
di questi buchi neri, in particolare al centro della nostra galassia.
I buchi neri primordiali sono di interesse per la cosmologia e potrebbero
offrire una conferma diretta di alcune previsioni sulla teoria quantistica dei
buchi neri 
(effetto Hawking), tuttavia non ne sono ancora stati osservati.

\section{Premesse}

\subsection{Calcolo ingenuo del raggio di Schwarzchild in gravitazione
   Newtoniana}

Se lanciamo un oggetto verso l'alto, tipicamente lo vediamo rallentare,
raggiunge un'altezza massima e poi ricadere a terra. Non sempre \`e cos\`{\i}.
Se la velocit\`a con cui l'oggetto \`e lanciato verso l'alto \`e
sufficientemente alta, l'oggetto potr\`a allontanarsi senza ricadere. Quanto
velocemente dobbiamo lanciarlo? La minima velocit\`a \`e chiamata ``velocit\`a
di fuga''. Non \`e difficile calcolare, nell'ambito della teoria Newtoniana
della gravitazione, quale sia la velocit\`a di fuga.

Uguagliando energia cinetica ed energia potenziale gravitazionale
\begin{dmath*}
   \frac{1}{2}mv^{2} = G\frac{mM}{R^{2}} 
\end{dmath*},
si trova
\begin{dmath*}
   v = \sqrt{ \frac{2GM}{R^{2}}} 
\end{dmath*}.
Si noti che la velocit\`a di fuga \`e indipendente dalla massa $m$.



\section{La soluzione di Schrwarzchild}
La metrica di Schwarzchild \`e
\begin{dmath*}
   \df[2]{s} = \left( 1 - \frac{2GM}{c^{2} R} \right) \df[2]{t} + \left( 1 -
      \frac{2GM}{c^{2}R} \right)^{-1} \df[2]{r} + R^{2} \left(
      \ud[2]{\vartheta} + \sin^{2} \vartheta \df[2]{\varphi} \right) 
\end{dmath*}.

\section{Buchi neri in astrofisica}


