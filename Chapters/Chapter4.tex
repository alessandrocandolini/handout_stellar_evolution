
%*******************************************************
% Chapter 4
%*******************************************************
\myChapter{Oltre il limite di Chandrasekhar}\label{chandra}
\minitoc\mtcskip
\noindent Nel 1930 il giovane fisico indiano Subramahyan Chandrasekhar decise di imbracarsi per l'Inghilterra con nel cuore il sogno di raggiungere l'Inghilterra dove studiare e diventare un giorno membro della Royal Society.
Durante il viaggio, Chandra si dedicò allo studio delle fasi terminali dell'evoluzione stellare.
\par
Egli in particolare si accorse che una nana bianca di massa superiore ad un certo valore critico stimato pari a circa $1,4 M_{\odot}$ non potr\`{a} mai essere gravitazionalmente stabile. Non possono semplicemente esistere nane bianche di massa maggiore a $1,4 M_{\odot}$; per simili valori di massa la forza di attrazione gravitazionale del corpo stellare su sé stesso risulta talmente intensa da vincere anche la repulsione tra gli elettroni degeneri.
\par
All'epoca purtroppo le sue idee furono duramente criticate, comunque a torto, dal fisico inglese Eddington, che non esitò a ironizzare in tono polemico sulle teorie di Chandrasekhar fino al punto di appellarsi all'esistenza di leggi di natura che avrebbero impedito ad una stella di comportarsi in maniere tanto assurde.
Per i suoi studi sull'evoluzione stellare Chadrasekhar ha ricevuto nel 1983 il premio Nobel. Il riconoscimento per l'importanza delle sue ricerche è arrivato oltre cinquant'anni dopo che i suoi risultati erano stati presentati. La vicenda rappresenta un triste promemoria di cosa possa accadere nella scienza quando certe persone sono a tal punto convinte delle proprie opinioni da non accettare il confronto con pareri diversi o contrastanti. Il fatto che ancora nel XX secolo si siano manifestati casicome questi, dove l'autorit\`{a} di uno scienziato abbia finito per ostacolare sul nascere i progressi della ricerca, è certamente un segnale grave; a parere strettamente personale dell'autore, il quale circa questi fatti non esita a dichiarare il proprio punto di vista, il comportamento di Eddington non può che definirsi sotto questo aspetto estremamente ascientifico. Solo misurando le proprie convinzioni con gli altri e aprendosi al dibattito con idee diverse dalle nostre si può avere la speranza di procedere di qualche passo nel cammino della ricerca scientifica.
\section{La pressione dello stato degenere in regime relativistico}
%\footnote{Sezione di approfondimento. Può essere tralasciata in prima lettura. Il contenuto delle sezioni successive non dipende per nulla da quanto qui esposto.}
\label{climit}
La pressione degli elettroni degeneri nell'approssimazione non relativistica dipende dalla densit\`{a} volumetrica di elettroni $n_{e}$ (sezione \ref{propdeg}). Più precisamente, se $P_{deg}$ è la pressione degli elettroni degeneri, dall'equazione (\ref{stato degenere}), si ha che:
\begin{equation}\label{prop53}
P_{deg} \propto n_{e}^{\frac{5}{3}}
\end{equation}
In condizioni di equilibrio idrostatico, la pressione $P_{g}$ che si esercita in un \emph{qualunque} punto all'interno del corpo stellare per effetto del peso della materia degli strati sovrastanti cresce, come ragionevole aspettarsi, con la massa della stella e con la sua densit\`{a}. Questo significa che, a parit\`{a} di densit\`{a}, più una stella è massiccia e più grande sar\`{a} la pressione che si eserciter\`{a} in un suo punto qualunque causa l'attrazione gravitazionale della massa stellare su sé stessa. Viceversa, a parit\`{a} di massa, la pressione sar\`{a} tanto più grande quanto più densa sar\`{a} la stella. Si può tradurre in forma quantitativa questo fatto scrivendo la relazione:
\begin{equation}\label{pmr}
P_{g}\propto M^{\frac{2}{3}}\rho^{\frac{4}{3}}
\end{equation}
dove $M$ e $\rho$ sono rispettivamente la massa e la densit\`{a} della stella. Si noti che la pressione $P_{g}$, come anche la pressione degli elettroni degeneri (\ref{stato degenere}), non dipende da quale punto del corpo stellare si scelga di considerare: la pressione ha lo stesso valore in ogni punto della stella. Non daremo qui una dimostrazione della (\ref{pmr}); l'equazione può essere dedotta dall'equazione per l'equilibrio idrostatico e facendo uso dell'equazione di stato del gas perfetto.
\par
Fatte queste due precisazioni, siamo pronti per affrontare il problema del limite di Chandrasekhar. In questa sezione si cercher\`{a} di sottolineare un fatto in particolar modo, e questo fatto è il seguente: finché si considera come formula per la pressione degli elettroni degeneri la (\ref{stato degenere}), ricavata nell'approssimazione non relativistica, non si arriver\`{a} mai a trovare qualche evidenza che esiste il limite di Chandrasekhar, e cioè che nane bianche di massa superiore a $1.4 M_{\odot}$ non possono stabilizzarsi in condizione di equilibrio idrostatico. Ma se si ricava una nuova espressione per la pressione degli elettroni degeneri nel caso relativistico, allora automaticamente emerge la problematica del limite di Chandrasekhar. La deduzione di questo fatto richiede ovviamente il ricorso a particolarismi tecnici e ad un certo formalismo matematico che esulano certo dagli obiettivi modesti del presente documento. Si vuole in questa sezione presentare qualche sommario argomento a favore di quanto sopra detto, senza pretesa di alcuna completezza. Questa sezione vuole giusto essere un assaggio che indichi al lettore dove nasca l'idea dell'esistenza del limite di Chandrasekhar. In ogni caso la la presente sezione è da considerarsi come materiale integrativo di approfondimento. Il lettore che lo volesse non tardi a proseguire nelle sezioni successive.
\par
Un calcolo preciso basato sulla statistica di Fermi--Dirac fornisce la seguente espressione per il calcolo della pressione degli elettroni degeneri nel caso relativistico:
\begin{equation}
P_{deg}=\frac{1}{4} \sqrt[3]{\left( \frac{3}{8\pi} \right)} \cdot h \cdot c\cdot n_{e}^{\frac{4}{3}}
\end{equation}
cioè:
\begin{equation}\label{prop43}
P_{deg} \propto n_{e}^{\frac{4}{3}}
\end{equation}
Può apparire una piccola differenza rispetto alla relazione di proporzionalit\`{a}(\ref{prop53}). La (\ref{prop43}) e la (\ref{prop53}) differiscono solo per la potenza con cui vi figura la densit\`{a} di elettroni $n_{e}$. Eppure proprio da questa apparentemente quasi insignificante differenza, che concettualmente non sembra nascondere niente di nuovo, tra le due relazioni prende avvio il problema del limite di Chandrasekhar.
Finchè si lavora con la relazione (\ref{prop53}), qualsiasi sia la pressione $P_{g}$ dovuta all'attrazione gravitazionale della stella su sé stessa, si può sempre scegliere $n_{e}$ abbastanza grande perch\'{e} risulti che $p_{g}$ e $P_{deg}$ si equilibrano a vicenda. Se una stella è molto massiccia, per la (\ref{pmr}) la pressione sar\`{a} molto grande; via via che la stella si contrae il suo volume diminuisce, e siccome la sua massa non cambia, \emph{sia} $n_{e}$ che $\rho$ crescono; allora la pressione $p_{g}$, in base alla (\ref{pmr}) cresce, perch\'{e} cresce $\rho$, e anche $P_{deg}$ cresce perch\'{e} cresce anche $n_{e}$; ma $P_{deg}$ cresce \underline{più rapidamente} di $P_{g}$, perch\'{e}, per la (\ref{prop53}), $P_{deg}$ cresce con potenza $5/3$ della densit\`{a}, mentre $P_{g}$ cresce con una potenza più piccola, $4/3$ appunto, della densit\`{a}. Ad un certo punto $P_{deg}$ sar\`{a} cresciuta abbastanza da equilibrare $p_{g}$.
In termini matematici si ha che il rapporto $P_{deg}/P_{g}$ cresce al crescere della densit\`{a}, cioè via via che il collasso della stella procede. Ad un certo punto questo rapporto sar\`{a} cresciuto al punto da valere esattamente $1$.
\par
Ma se invece della (\ref{prop53}) usiamo la (\ref{prop43}) questo non è più vero, perch\'{e} in questo caso si avrebbe che sia $P_{deg}$ sia $P_{g}$ crescono con la medesima potenza della densit\`{a}, la potenza $4/3$: via via che la stella si contrae cioè aumentano nella stella maniera sia la pressione di degenerazione sia la pressione dovuta all'attrazione gravitazionale della stella su sé stessa. non è più come nel caso precedente che le due pressioni crescono in maniera diversa via via che la stella prosegue nella fase di collasso. In termini matematici il rapporto $P_{deg}/P_{g}$ non dipende più da una qualche potenza della densit\`{a}, ma dipende solo dalla massa della stella. Vale la relazione:
\begin{equation}
\frac{P_{g}}{P_{deg}}\propto M^{\frac{2}{3}}
\end{equation}
Per un certo valore critico della massa della stella, la pressione degli elettroni degeneri non sar\`{a} mai in grado di eguagliare la pressione dovuta all'attrazione gravitazionale della stella su sé stessa, e conseguentemente la pressione degli elettroni degeneri non sar\`{a} sufficiente ad arrestare il collasso gravitazionale di queste stelle. Stelle troppo massicce non potranno stabilizzarsi nello stadio di nana bianca. Per stelle degeneri del tutto prive di idrogeno il valore critico per cui questo succede vale:
\begin{displaymath}
M_{c} =1.44 M_{\odot}
\end{displaymath}
Questo limite è il limite di Chandrasekhar e la massa critica $M_{c}$ prende il nome di massa di Chandrasekhar.
\section{Pressione dei neutroni degeneri: le stelle di neutroni}\label{stelle neutroni}
Siamo finalmente pronti per affrontare il grande quesito lasciato aperto alla sessione precedente, e cioè cosa è del nocciolo di una gigante rossa massiccia una volta che quest'ultima è esplosa come supernova? 
\par
Riprendiamo brevemente quello che abbiamo visto sulle fasi terminali di sviluppo delle stelle.
Sappiamo che stelle di massa \emph{complessiva} inferiore ad $8M_{\odot}$ terminano la loro esistenza nello stadio di nana bianca in continuo, anche se lento, procedere verso una sicura morte termica; ci sono stelle (quelle di massa inferiore a $o.5 M_{\odot}$) che si stabilizzano a nane bianche prima ancora di innescare il bruciamento dell'elio, altre più massicce che attraversano invece una ulteriore fase di stabilit\`{a}, garantita dalla reazioni di fusione dell'elio nel nocciolo, prima di evolvere a nane bianche. Comunque sia, passando o no per la combustione dell'elio, tutte queste stelle concludono la loro esistenza come nane bianche. Le nane bianche sono in equilibrio idrostatico perch\'{e} la pressione di degenerazione degli elettroni è sufficiente a sostenere la stella da un ulteriore collasso.
\par
Più problematica è la fine di stelle di massa superiore a $8M_{\odot}$. Queste escono dalla fase di giganti rosse in modo esplosivo come supernovae. Nel fenomeno della supernova la stella espelle in maniera violenta gli strati più esterni, che andranno a formare la nebulosa residuale (da non confondere con le nebulose planetarie!). Proprio a questo punto si inserisce la domanda: ``che è del nocciolo?'' Eh, se lavorassimo con l'approssimazione non relativistica della pressione degli elettroni degeneri la conclusione sarebbe sempre la stessa: contraendosi per effetto della gravit\`{a}, ad un certo punto il nocciolo finir\`{a} per stabilizzarsi; la pressione degli elettroni degeneri a questo punto sar\`{a} in ogni caso in grado di sorreggere la stella da un ulteriore collasso e il nocciolo, tutto ciò che rimane della stella esplosa, terminer\`{a} la sua esistenza anch'esso nello stadio di nana bianca. Ma purtroppo per noi le cose non sempre vanno cos\`{\i}{} facili come si vorrebbe e le complicazioni non tardano ad arrivare. Basta riscrivere la formula per la pressione degli elettroni degeneri nel caso relativistico per giungere, con qualche conticino, all'inevitabile conclusione che non potr\`{a} mai esserci una nana bianca gravitazionalmente stabile avente massa superiore a circa $1.4 M_{\odot}$. In genere la massa residua del nocciolo di una supernova ha massa superiore a questo limite. \`E ragionevole aspettarsi che gli sconvolgimenti durante l'esplosione della supernova coinvolgano in qualche misura anche il nocciolo, per cui esso può perdere almeno in parte la sua massa originaria e rientrare al di sotto del limite critico di $1.4 M_{\odot}$. Non si può escludere che processi di questo tipo avvengano, e cioè che il nocciolo di una gigante rossa massiccia, nocciolo la cui massa è senz'altro maggiore di $1.4 M_{\odot}$, possa perdere una frazione della propria massa nell'esplosione della stella come supernova, e rientrare entro il fatidico limite di $1.4 M_{\odot}$. Se questo succede allora il nocciolo finir\`{a} per stabilizzarsi nello stadio di nana bianca.
Ma questo non può succedere sempre. Ci saranno delle volte in cui questo anche accade, ma più in generale il nocciolo, che durante la permanenza della stella nello stadio di gigante rossa massiccia aveva massa senza dubbio superiore al limite di Chandrasekhar, si ritrova, dopo l'esplosione, ancora con una massa superiore a $1.4 M_{\odot}$. Questo nocciolo semplicemente \emph{non può} stabilizzarsi come nana bianca: la pressione degli elettroni degeneri non riuscir\`{a} \emph{mai} a diventare abbastanza intensa per controbilanciare la tendenza della stella a contrarsi ulteriormente. Non ci sono speranze che il nocciolo di massa maggiore a $1.4 M_{\odot}$ raggiunga infine una condizione di equilibrio idrostatico come nana bianca; la pressione degli elettroni degeneri non crescer\`{a} mai abbastanza da sostenere il ``peso'' della stella.
\par
Lo stadio di nana bianca viene superato e la stella prosegue apparentemente in modo inarrestabile la sua corsa verso il collasso gravitazionale. La pressione degli elettroni degeneri non basta a frenare il collasso, e l'attrazione gravitazionale dell'astro su sé stesso ha il sopravvento sulla pressione di degenerazione. Cosa succede a questo punto?
Per descrivere quanto avviene riportiamo le testuali parole di Leonida Rosino:``I nuclei, tremendamente compressi, si spezzano, si fondono insieme con gli elettroni che non riescono a equilibrare il peso che li schiaccia, si trasformano, per l'annullamento delle cariche opposte, in neutroni'' \Cite{rosino}, p. 806.
\par
Frasi di questo tipo si trovano frequentemente riportate nei testi di divulgazione. L'autore rinuncia dal tentativo di fornire una qualche spiegazione di cosa si intenda con ``fusione dei protoni e degli elettroni a formare neutroni'' e si limita a constatare il fatto che, una volta avvenuta, in qualche modo, questa ``fusione'', quello che si origina è una stella di neutroni. Non è detto che la stella di neutroni formi un sistema stabile, e su questo punto torneremo tra breve. ma intanto sicuramente si è formato un oggetto composto di soli neutroni, forse gravitazionalmente stabile o forse ancora in collasso.
I neutroni che si sono cos\`{\i}{} formati sono neutroni stabili. In condizioni operative ordinarie il neutrone libero è una particella instabile con vita media di circa un quarto d'ora. 
\par
%\`E noto che il volume di un atomo sia dato dagli orbitali elettronici (i quali in compenso hanno un ruolo marginale nel determinare la massa complessiva di un atomo). le dimensioni dell'edificio atomico sono individuate dagli elettroni che orbitano attorno al nucleo. Per il resto, ad eccezione del nucleo centrale, l'atomo è, come si potrebbe dire, vuoto\footnote{Termine improprio perch\'{e} non tiene conto delle fluttuazioni quantistiche del vuoto dovute all'indeterminazione nella misura dell'energia. Sebbene il termine sia usato impropriamente, non essendo l'autore al momento in grado di ricorrere ad un termine più rigoroso, confida sul buon senso del lettore che abbia efficacemente colto il significato della proposizione.}. Nel momento in cui gli elettroni e i protoni, in qualche modo, si ``fondono'', lo spazio dell'edificio atomico prima non occupato da alcuna particella reale ora si presenta riempito da neutroni.
L'ipotesi dell'esistenza di stelle neutroniche fu avanzata da Landau nel 1932, ed in seguito ripresa da Volkoff e Oppenheimer. Simili oggetti, proprio perch\'{e} costituiti da soli neutroni quasi in contatto diretto, risultano estremamente densi. Si stima che la densit\`{a} di una stella di neutroni sia dell'ordine di $10^{18}$ Kg/m$^{3}$, cioè 15 ordini di grandezza più grande della densit\`{a} dell'acqua (che è $10^{3}$ kg/m$^{3}$).
%Le propriet\`{a} fisiche che ci si aspetta dalla materia neutronica che costituisce le stelle di neutroni sono molto interessanti, e su di esse torneremo tra breve. Vediamo prima di chiarire se una stella di neutroni è o meno un oggetto gravitazionalmente stabile. 
Se la massa in gioco non è superiore a circa $3$ o $4$  masse solari, le forze di repulsione tra neutroni \emph{degeneri} sono sufficienti a controbilanciare la tendenza della stella a contrarsi ulteriormente. Per valori di massa pari a 3 o 4 volte la massa del Sole il nocciolo in collasso gravitazionale trova la sua condizione di equilibrio idrostatico nella stella di neutroni.
\par
Le propriet\`{a} fisiche che ci si aspetta dalla materia neutronica che costituisce le stelle di neutroni sono molto interessanti. I neutroni si dispongono ai vertici di un reticolo cristallino: per questa ragione si ritiene che una stella di neutroni non sia interamente gassosa, nonostante l'elevatissima temperatura che la caratterizza (dell'ordine di qualche centinaio di milioni di gradi) ma possegga uno spesso involucro esterno solido (nel senso di cristallino); solo vari kilometri più in profondit\`{a}, via via che dalla superficie esterna ci si porta verso il nucleo centrale, la temperatura diventa cos\`{\i}{} elevata da impedire ai neutroni di disporsi ordinatamente in uno schema geometrico  di tipo cristallino. Nelle regioni interne, oltre ai neutroni, è ancora possibile trovare in piccole percentuali protoni, elettroni  e alcune altre particelle più esotiche (mesoni ecc.). L'intero edificio è avvolto da un sottilissimo strato di gas formato da nuclei atomici ed elettroni; lo spessore di questo strato più esterno pare non superiore ad un metro \Cite{rosino}.
\par
Se la massa in gioco è superiore a 3 o 4 masse solari (la cosiddetta \emph{massa di Oppenheimer}) il nocciolo in contrazione non può stabilizzarsi neppure nello stadio di stella di neutroni: la pressione tra neutroni degeneri non è sufficiente ad impedire un ulteriore collasso del corpo stellare. Cosa sar\`{a} del corpo stellare in questo caso è un problema che sar\`{a} accennato nella sezione conclusiva.
\section{Pulsar}\label{pulsar}
\subsection{La scoperta delle pulsar}\label{scoperta pulsar}
Quello che si è descritto, in termini piuttosto generali e discorsivi alla sezione \ref{stelle neutroni} è un modello \emph{teorico} di come il nocciolo residuo di una supernova possa evolvere, se avente massa compresa indicativamente tra $1.4 M_{\odot}$ e $4M_{\odot}$, a formare una stella di neutroni. Le propriet\`{a} e la struttura di questi oggetti sono frutto di elaborazioni teoriche. Che prove sperimentali si hanno che le cose vadano proprio come si è detto alla sezione \ref{stelle neutroni}, e che invece il modello proposto non sia erroneo? Ci sono cioè dati osservazionali che confermano prima di tutto l'esistenza di stelle di neutroni, ed in secondo luogo che, nell'eventualit\`{a} esistano, questi oggetti abbiano effettivamente le propriet\`{a} teoricamente previste?
\par
Ipotizzate teoricamente ancora nel 1932 come si è visto, le stelle di neutroni sembrano aver ricevuto una conferma sperimentale piuttosto forte a favore della loro effettiva esistenza a seguito della scoperta nel 1967, ad opera dell'allora dottoranda a  Cambridge Jocelyn Bell , delle cosiddette \emph{pulsar}, acronimo inglese di \emph{pulsating radio source}.
\par
Vediamo di cosa si tratta. Come tutte le storie, anche questa ha un inizio, che si può far risalire alla tarda estate dell'ormai lontano 1967, a Cambridge.\footnote{L'esposizione è ispirata all'eccellente resoconto di Piero Tempesti in \Cite{tempesti}.} Un radiotelescopio, costituito da un insieme di tralicci sostenente duemila piccole antenne, sta scandagliando la volta stellata alla lunghezza d'onda di 3.68 metri. Lo strumento permette di registrare una serie di impulsi brevissimi, di pochi centesimi di secondo, separati da un intervallo regolare di 1,3 secondi. Di cosa si poteva trattare? Jocelyn Bell, nota l'estrema regolarit\`{a} con cui eventi simili si ripetono: in particolare questo tipo di segnale si ritrova in giorni diversi alla stessa ora siderale, e cioè quando transita sul meridiano locale una stessa regione di cielo che Bell identifica appartenere alla costellazione della Piccola Volpe. La studentessa decide di parlarne con il suo professore, Antony Hewish, e insieme decidono di fare una registrazione ad alta velocit\`{a} all'ora siderale in cui ci si aspetta di ricevere questi impulsi. Per settimane Jocelyn si impegna tutte le sere (in autunno la Volpetta transita in meridiano verso sera) a registrare il segnale radio raccolto dal radiotelescopio, ma invano: gli impulsi brevissimi sembrano scomparsi, al punto da far ritenere che quei segnali precedentemente registrati fossero in realt\`{a} rumore spurio di chiss\`{a} quale origine. Ma Jocelyn non si arrende. Riportiamo le parole con cui Piero Tempesti a questo punto narra i fatti accaduti \Cite{tempesti}:
\begin{quote}
`` Ma Jocelyn ha quel pensiero fisso che non la abbandona, e una sera, all'ora giusta, torna di nuovo davanti alla consolle dello strumento. Aziona il commutatore di velocit\`{a}\footnote{All'epoca il segnale raccolto dal radiotelescopio veniva registrato da un pennino su un rullo di carta scorrevole. Nelle indagini di Jocelyn si richiedevano registrazioni ad elevata velocit\`{a}, cos\`{\i}{} che fosse possibile esaminare l'andamento del segnale su scale più ampie.} ed attende: un'intelligenza, un cuore ed un grande orecchio metallico tesi verso il cielo per riconoscere, fra i mille tenuissimi «rumori» che vengono percepiti, un evanescente segnale proveniente da un ignoto mondo delle sconfinate profondit\`{a} siderali. l'occhio ansioso segue il pennino che traccia sulla carta che scorre le solite minutissime fluttuazioni del «rumore di fondo»; passa un minuto, due, e non succede nulla; ormai il momento buono sta per finire e la ragazza comincia a pensare che la sua è un'ostinazione inutile e che hanno ragione tutti a dire di non pensarci più\ldots ma improvvisamente ecco che il pennino effettua una rapida oscillazione, si stabilizza un istante e ne compie una seconda, e poi un'altra e poi un'altra ancora\ldots''
\end{quote}
A questo punto Jocelyn si rivolge a Hewish, il quale ritiene però che i segnali captati, vista la cadenza estremamente regolare con cui si ripresentano ogni 1,3 secondi, ciascuno con la medesima durata,  siano segnali artificiali di origine terrestre. Poche settimane più tardi Jocelyn scopre un'altra serie di impulsi, questa volta separati tra loro da intervalli regolari di 1,2 secondi, che si verificano quando il Leone transita in meridiano.
\par
La notizia viene diramata dalla rivista scientifica \emph{Nature} il 24 febbraio del 1968. Gli impulsi sembrano provenire da regioni di cielo estremamente piccole, praticamente puntiformi. Anche se all'epoca non è ancora ben chiaro il meccanismo con cui simili segnali si generano, le sorgenti vengono battezzate con il nome di pulsar.
\par
Ad Antony Hewish fu assegnato il premio Nobel per la fisica nel 1974, mentre Jocelyn non fu neppure nominata. Solo anni dopo ebbe alcuni riconoscimenti di minor prestigio \citep{hack}.
\subsection{Il modello Pacini--Gold}\label{faro}
Il modello oggi più accreditato che fornire una spiegazione dell'emissione delle pulsar è il cosiddetto modello Pacini--Gold, dal nome di Franco Pacini che per primo lo propose, nel 1968, assieme con Thomas Gold. il modello è meglio conosciuto come modello ``faro''.
\par
Una stella comunemente genera nello spazio un campo magnetico. Sulle sorgenti di un campo magnetico stellare non ci dilungheremo in questa sede.
Durante la fase di collasso, il campo magnetico eventualmente  generato dal nocciolo residuo dell'esplosione di supernova aumenta notevolmente di intensit\`{a}. Sussiste una relazione di proporzionalit\`{a} inversa tra superficie del corpo stellare e intensit\`{a} del campo magnetico \citep{hack}. Le ragioni di questo fatto non saranno qui chiarite.
Sempre a causa della contrazione, la velocit\`{a} angolare $\omega$ del corpo stellare aumenta, non diversamente da quanto descritto per il processo di formazione stellare (sezione \ref{contrazione}). Questo effetto è spiegabile ammettendo il principio di conservazione del momento della quantit\`{a} di moto.
Gli elettroni liberi che ancora si trovano in moto in prossimit\`{a} della superficie della stella (se ancora si può parlare di stella) si trovano in moto a velocit\`{a} lineari praticamente elevatissime, relativistiche, velocit\`{a} rispetto le quali la velocit\`{a} della luce nel vuoto $c$ non è più trascurabile.Come tutte le particelle cariche in moto in un campo magnetico, anche gli elettroni vengono accelerati per effetto della componente magnetica della forza di Lorentz\footnote{Una particella elettricamente carica, di carica elettrica $q$, in moto con velocit\`{a} $\vec{v}$ in un campo magnetico di induzione magnetica $\vec{B}$, subisce una forza $\vec{F}$, detta componente magnetica della forza di Lorentz, data da:
\begin{displaymath}
\vec{F}=q \vec{v} \times \vec{B}
\end{displaymath}
dove il simbolo $\times$ è usato per indicare il prodotto vettoriale.}.
A causa del moto accelerato, gli elettroni, in moto a velocit\`{a} relativistiche, emettono radiazione (\emph{radiazione di sincrotrone}) che è funzione della lorocarica e velocit\`{a} (prossima, meglio ripetere, alla velocit\`{a} della luce nel vuoto), nonché dell'intensit\`{a} del campo magnetico.
Diversamente da un'emissione di tipo termico, generalmente isotropa, l'emissione di sincrotrone risulta emessa lungo direzioni privilegiate, nel nostro caso specifico  lungo uno stretto cono sopra e sotto i poli magnetici (laddove cioè le linee di campo magnetico vanno chiudendosi).
\par
Se l'asse del campo magnetico e l'asse di rotazione del corpo stellare sono inclinati l'uno rispetto all'altro, può avvenire che , a ogni rotazione della stella, uno dei due getti di radiazione venga a trovarsi rivolto verso la Terra.
%\begin{figure}
%\includegraphics[width=15cm]{evoluzione_stellare0}
%\caption{Modello Pacini--Gold di pulsar \Cite{dario}.}
%\end{figure}
L'energia che gli elettroni irradiano va a spese dell'energia rotazionale del pulsar, che al trascorrere del tempo, diminuir\`{a} la propria velocit\`{a} angolare\footnote{L'energia cinetica rotazionale di un corpo dipende dalla velocit\`{a} angolare $\omega$, in particolare è data dalla relazione $E_{rot}=\frac{1}{2}I\omega^{2}$, con $I$ momento di inerzia del corpo.
Se $E_{rot}$ diminuisce, se non può variare $I$ ($I$ dipende dalla geometria e dalla massa del corpo), dovr\`{a} diminuire $\omega$.} $\omega$, ovvero allungher\`{a} il proprio periodo \Cite{rosino}.
\par
Il lettore non confonda la descrizione sommaria e per certi tratti grossolana che si è data del meccanismo di emissione dei pulsar con una trattazione rigorosa e approfondita del modello Pacini--Gold. Quello che qui si è cercato di fare è dare al lettore un'idea generale di come le stelle di neutroni permettano di spiegare, per via teorica, le propriet\`{a} osservate dei pulsar.
\par
Un ulteriore appoggio alla teoria secondo cui i pulsar altro non sono che stelle di neutroni è rappresentato dalla presenza di un pulsar all'interno della Crab Nebula che, come si è detto, è un resto di supernova. Un altro pulsar, la Vela X, fu scoperto nell'omonima costellazione anch'esso associata al resto di una supernova.



