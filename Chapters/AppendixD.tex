%*******************************************************
% Appendix D
%*******************************************************

\myChapter{Cenni sulla teoria quantistica dei buchi neri}\label{app:buchi neri}
\minitoc\mtcskip
L'esistenza di regioni di spazio-tempo \emph{dalle} quali \`e preclusa la trasmissione di ogni genere di segnale o informazione verso un possibile osservatore esterno \`e una conseguenza deducibile nell'ambito di una teoria classica della gravitazione ed in ultima analisi essa pu\`o essere fatta  discendere dall'essere l'interazione gravitazionale sempre attrattiva, perlomeno in situazioni ordinarie.
Per designare siffatte regioni di universo, inacessibili all'osservazione, il fisico americano John A. Wheeler ha coniato, nell'autunno del 1967, il termine, all'epoca quanto mai appropriato, di \emph{buco nero}.
\par
Modelli di buchi neri, messi a punto nell'ultimo trentennio tenendo conto di taluni effetti di natura genuinamente quantistica, paiono tuttavia suggerire l'eventualit\`a che lo scenario classico poc'anzi tratteggiato possa non essere considerato del tutto corretto.
Gi\`{a} nel 1974 Stephen Hawking, compiendo un'analisi del comportamento della materia in prossimit\`a di un buco nero alla luce di certe considerazioni derivabili nell'ambito delle teorie quantistiche dei campi, dedusse per via teorica come i buchi neri vadano incontro ad emissione termica.
%Sviluppi sotto il profilo teorico si sono avuti recentemente mediante un approccio basato sulla teoria delle stringhe e paiono confermare i risultati dedotti da Hawking.
\par
Per ragioni che saranno meglio analizzate nel seguito, una verifica operativa del fatto che i buchi neri siano soggetti ad irragiamento cade oltre le effettive possibilit\`a di sperimentazione, e pertanto l'ipotesi che i buchi neri emettano non \`e, almeno attualmente, suscettibile di alcuna conferma empirica diretta.
La discussione in merito, che ad oggi non ha conosciuto una risposta in ogni
modo definitiva, si inserisce nel quadro pi\`u ampio dell'elaborazione di una
teoria coerente della gravitazione su scala quantistica capace di risolvere
l'incompatibilit\`a tra due delle massime formulazioni teoriche del Novecento: la meccanica quantistica e la teoria generale della relativit\`a.
Una simile impresa, lungi dall'essere stata condotta a termine con successo, rappresenta quasi certamente una delle frontiere pi\`u ambite della fisica di questo secolo.
\section{Soluzione di Schwarzschild delle equazioni di campo}
Argomenti a sostegno della tesi circa la presenza nell'Universo di regioni dalle quali neanche i segnali luminosi possono evadere, confinati indefinitamente entro le medesime dagli intensi campi gravitazionali in gioco, possono in qualche modo farsi risalire, nell'ambito della legge newtoniana di gravitazione universale, ai lavori pioneristici che John Mitchell (1724-1793) e Pierre Simon de Laplace (1749-1827) diedero alla luce sul finire del Settecento.
\par
\`E tuttavia solo con l'elaborazione, ad opera di Albert Einstein nel 1915, di una teoria generalizzata della gravitazione (la cosiddetta relativit\`a generale), atta a descrivere in maniera precisa, e per quanto se ne sa, fondamentalmente corretta la propagazione della radiazione elettromagnetica in regimi a gravit\`a forte, che si ebbero a disposizione gli strumenti teorici e un apparato matematico necesari ad una accurata analisi dei buchi neri.
A pochi mesi dall'originario lavoro di Einstein sulla relativit\`a generale,
l'astronomo Karl Schwarzchild per primo ricav\`o una soluzione delle equazioni
di campo della nuova teoria (le cosiddette equazioni di Einstein), soluzione che
\`e  da considerarsi


\section{La radiazione di Hawking}





