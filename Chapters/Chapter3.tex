
%*******************************************************
% Chapter 3
%*******************************************************

\myChapter{L'esodo dalla sequenza principale}\label{esodo}
\minitoc\mtcskip

\section{Premesse}\label{Premesse}
La fase di stabilit\`{a} descritta in sezione \ref{stabilita} viene definitivamente meno nel momento in cui il ``combustibile'' del nucleo comincia ad esaurirsi.
Una volta che le scorte di idrogeno disponibile nel nocciolo si fanno via via sempre più scarse, le reazioni termonucleari procedono a ritmi decisamente insufficienti a garantire un tasso di produzione di energia costante e ottimale al mantenimento delle condizioni di equilibrio della stella.
Come conseguenza di questo fatto, la stella va incontro a rilevanti modificazioni  nell'organizzazione della propria struttura; in genere si assiste ad un cospicuo ridimensionamento delle dimensioni del nucleo centrale della stella.
Il nocciolo della stella attraversa una fase di contrazione a seguito della quale, per ragioni del tutto analoghe a quelle gi\`{a} discusse per il processo di formazione stellare (sezione \ref{contrazione}), aumenta la propria temperatura. 
In accordo con questi cambiamenti strutturali, anche i parametri che caratterizzano una stella subiranno delle variazioni. Complessivamente, si designa questa fase di sviluppo della stella come \emph{esodo dalla sequenza principale} \Cite{rosino}.
\par
Ad ogni modo, il comportamento successivo della stella presenta, nello specifico, tratti diversificati a seconda che si prenda in esame stelle massicce o stelle meno massicce.
Come gi\`{a} si è avuto modo di verificare durante la fase di stabilit\`{a}, la configurazione del tracciato evolutivo di una stella dipende fortemente dai valori di massa in gioco.
Sar\`{a} conveniente esaminare separatamente tre distinti casi: stelle di massa complessiva pari a qualche decimo della massa solare (diciamo aventi massa inferiore a circa $0.5 M_{\odot}$); stelle di massa complessiva inferiore a $8 M_{\odot}$; stelle di massa complessiva maggiore di $8 M_{\odot}$.
\section{Stelle di massa $M<0.5M_{\odot}$}
Sia dato il caso di stelle aventi massa complessiva pari a qualche decimo della massa solare. Come parametro di riferimento, stabiliremo di considerare appartenenti a questa categoria tutte quelle stelle aventi indicativamente massa inferiore diciamo a $0.5 M_{\odot}$.
\footnote{M. Crippa, M. Fiorani, \emph{Geografia Generale}, Arnoldo Mondadori Scuola, Milano 2002, p. 34}
\par
Si tratta di stelle di bassa sequenza principale, la cui fase di stabilit\`{a} ha i caratteri descritti in sezione \ref{nane rosse}. Cosa succede a queste stelle quando le riserve di idrogeno nel nocciolo cominciano ad esaurirsi? Si è gi\`{a} visto sopra (sezione \ref {Premesse}) che, come accade per \emph{tutte} le stelle in esodo dalla sequenza principale, indipendentemente da quale sia la loro massa, anche per queste stelle si assiste ad una contrazione del nocciolo centrale.
Nel momento in cui la quantit\`{a} di energia $\Delta E$ persa dalla stella per irraggiamento non è più compensata dall'energia prodotta dalle reazioni nucleari nel nocciolo (e ciò in conseguenza della mancanza di sufficienti quantit\`{a} di idrogeno nel nocciolo stesso), le regioni centrali della stella vanno incontro a collasso gravitazionale. Il fenomeno avviene per ragioni del tutto analoghe a quelle descritte nel processo di formazione stellare (sezione \ref{contrazione}).
\par
La contrazione gravitazionale del nocciolo è accompagnata da un corrispondente aumento di temperatura, anche qui non diversamente da quanto avviene nel processo di formazione stellare (sezione \ref{aumentot})
\par
Da calcoli basati sulla teoria quantistica, si stima che temperature maggiori di 100 milioni di kelvin \Cite{kittel} siano tali da consentire il verificarsi di reazioni di fusione nucleare tra nuclei di elio $^{4}He$. \footnote{Si torner\`{a} su questo punto in sezione \ref{giganti rosse}, parlando delle giganti rosse}
\par
Si potrebbe allora pensare che il collasso gravitazionale del nocciolo, con corrispondente aumento della temperatura centrale, porti infine \emph{tutte} stelle ad innescare le reazioni di fusione dell'elio. In una teoria \emph{classica} dell'evoluzione stellare questo risultato sarebbe fondamentalmente corretto: prima o poi ogni stella in esodo dalla sequenza principale raggiungerebbe quel centinaio di milioni di kelvin indispensabili all'attivazione della fusione dell'elio, basterebbe solo aspettare che il nocciolo si contragga abbastanza.
\par
Questo fatto non è però vero! Sono proprio le stelle aventi massa $M < 0.5 M_{\odot}$ quelle per cui il nucleo non raggiunge \emph{mai} temperature sufficienti per innescare la fusione dell'elio.
Stelle di massa $M < 0.5 M_{\odot}$ non raggiungono mai, nella fase di collasso del nocciolo, quei valori di temperatura al centro indispensabili affinché avvengano le reazioni di fusione dell'elio.
Per queste stelle il collasso gravitazionale del nocciolo viene arrestato prima che la temperatura sia sufficientemente elevata. L'arresto è operato da forze di repulsione tra gli elettroni del plasma stellare. Queste forze sono di natura genuinamente \emph{quantistica}. Si parla di pressione degli elettroni degeneri (l'argomento sar\`{a} trattato in sezione \ref{degenere}). La forza di pressione dovuta agli elettroni degeneri è infine sufficiente a bilanciare la forza di attrazione gravitazionale che la stella esercita su sé stessa e ad impedire un'ulteriore contrazione del nocciolo centrale.
Detto in altre parole ancora, a causa di effetti \emph{tipicamente quantistici}, il collasso del nocciolo di stelle di massa $M<0.5M_{\odot}$ si arresta prima che la temperatura al centro della stella abbia raggiunto quei 100 milioni di kelvin necessari all'innesco della fusione nell'elio.
\par
A questo punto la stella si stabilizza (non potendo il suo nocciolo collassare ulteriormente) in una condizione di equilibrio \emph{idrostatico}.
Tuttavia, non disponendo di alcuna fonte di energia interna ad eccezione dell'energia termica, la stella, via via che irradia energia nello spazio, andr\`{a} raffreddandosi fino a diventare un piccolo oggetto spento.
La stella viene classificata come \emph{nana bianca}.
Cessata la contrazione, esaurita ogni riserva di combustibile nucleare, la nana bianca procede inesorabile nel suo cammino verso una sicura morte termica.
\par
Per la legge di Stefan--Boltzmann, la quantit\`{a} di energia $\Delta E$ persa per irraggiamento, nell'unit\`{a} di tempo, da un generico corpo è legata alla temperatura assoluta $T$ e alla superficie emettente $S$ del corpo stesso dalla relazione:
\begin{equation}
\Delta E = e \sigma S T^{4}
\end{equation}
dove $\sigma$ è la costante di Stefan pari a $\sigma=\mathnormal{5.68 \times 10^{-8}}$ W/(m$^{2}\cdot$ k$^{4}$) ed $e$ è il coefficiente di emissione (emettanza) compreso tra 0 e 1. Per un corpo nero $e=1$ \Cite{caf}.
\par
Essendo la superficie emettente della stella diventata, a seguito della fase di contrazione del nocciolo, estremamente piccola, la stella irradier\`{a} ben poca energia, e ne irradier\`{a} sempre di meno via via che la sua temperatura andr\`{a} diminuendo.
Il processo di raffreddamento risulter\`{a} conseguentemente lunghissimo.
Al suo termine, la stella cessa definitivamente ogni processo energetico per diventare un corpo freddo chiamato alle volte col nome di \emph{nana nera} \Cite{caf}.
Una stima del tempo di raffreddamento di una nana bianca è fornita ad esempio in \Cite{rosino}.
Dal momento che l'et\`{a} dell'universo, secondo le stime più attuali, è di circa 13 miliardi di anni, è presumibile che nessuna nana bianca abbia ancora raggiunto la morte termica.
\section{La pressione dello stato degenere}\label{degenere}
\subsection{Modello di calcolo della pressione degli elettroni degeneri}
Viene presentato nel seguito un modello semplificato di calcolo per la pressione degli elettroni degeneri. Lo scopo è quello di fornire al lettore da un lato un'evidenza della natura genuinamente quantistica della pressione dello stato degenere, dall'altra un'utile strumento per comprendere meglio da quali fattori, e da quali no, dipenda questa pressione. Soprattutto quest'ultimo punto sar\`{a} di importanza decisiva nel seguito. Il lettore non abbia la pretesa che poche righe bastino ad esaurire compiutamente che cosa si intenda per \emph{modello} in fisica. Quello che qui preme sottolineare è che un modello muove da ipotesi di lavoro anche piuttosto restrittive: la validit\`{a} di queste e, più in generale, dell'intero modello, è garantita solamente dalla ``bont\`{a}'' (\Cite{caf}, p. D52) dei risultati.
Premettiamo sin d'ora che la deduzione rigorosamente esatta del calcolo della pressione di degenerazione va fatta nell'ambito della distribuzione statistica di Fermi--Dirac.
Il risultato esatto cui si perviene è il seguente: indicando con $P_{deg}$ la pressione degli elettroni degeneri, con $m_{e}$ la massa dell'elettrone, con $n_{e}$ il numero di elettroni su unit\`{a} di volume e con $h$ la costante di Planck, si ha:
\begin{equation}\label{pesatta}
P_{deg}=\frac{1}{5} \sqrt[3]{\left( \frac{3}{8\pi} \right) ^{2}} \ \cdot \ \frac{h^{2}n_{e}^{\frac{5}{3}}}{m_{e}}
\end{equation}
Il modello di calcolo che qui sar\`{a} presentato risulta in accordo, a meno di coefficienti numerici,  con questo risultato.
\par

\smallskip
Si procede nella presentazione del modello.
Il lettore non si faccia spaventare per nessuna ragione dalle formule che può aver intravvisto nel seguito. Moltiplicazioni, divisioni e una radice cubica: ecco tutte le nozioni matematiche richieste, nulla di più.
Per una maggiore efficacia espositiva si proceder\`{a} scandendo ogni singolo passaggio del calcolo.
\begin{enumerate}
\item
Primo passaggio. Risulta conveniente introdurre una nuova grandezza, la \emph{densit\`{a} volumetrica di elettroni} $n_{e}$ cos\`{\i}{} definita: si consideri un volumetto cubico di lato $l$ e volume $V=l^{3}$; esso contiene un certo numero $N_{e}$ di elettroni \emph{liberi}. Allora $n_{e}$ è semplicemente il numero di elettroni per unit\`{a} di volume, cioè:
\begin{equation}\label{densita_elettroni}
n_{e}=\frac{N_{e}}{l^{3}}
\end{equation}
Fin qua nulla di particolarmente impegnativo, si è solo voluto introdurre per comodit\`{a} una nuova grandezza che ci dice, dato un certo volume, quanti elettroni liberi si trovano dentro questo volume. \`E solo questione di comodit\`{a} in seguito, tutto qua, si poteva anche fare a meno di introdurre, risparmiando una fatica ora ma più avanti ci saremo trovati a dover lavorare con due lettere ($N_{e}$ e $l^{3}$) invece che solo con una ($n_{e}$)!
\item
Secondo passaggio (che nulla ha a che fare col primo): scriviamo la relazione di indeterminazione:
\begin{equation}\label{Heisenberg}
\Delta x \Delta p > \frac{\hbar}{2}
\end{equation}
dove $\Delta x$ è l'\emph{indeterminazione} sulla posizione, $\Delta p$ l'\emph{indeterminazione} sulla quantit\`{a} di moto $p$ e $\hbar=h/2\pi$. La (\ref{Heisenberg}) è solo una delle relazioni di indeterminazione, peraltro anche abbastanza nota, che si possono far discendere dal principio di indeterminazione di Heisenberg. 
\par
Si noti che la relazione (\ref{Heisenberg}) è tipicamente quantistica: in meccanica   classica non vi è alcuna limitazione, almeno in linea di principio, alla precisione di due misure simultanee di posizione e quantit\`{a} di moto. 
\item
Terzo passaggio. Ci chiediamo: ''noto $n_{e}$, per un singolo elettrone quanto spazio è disponibile? ''
\par
Per rispondere è sufficiente porre il numero di elettroni $N_{e}$ pari a 1 (un singolo elettrone appunto) nella relazione (\ref{densita_elettroni}) e ricavare $l^{3}$:
\begin{equation}
n_{e}=\frac{1}{l^{3}}
\end{equation}
cioè:
\begin{equation}\label{lato}
l=\sqrt[3]{\frac{1}{n_{e}}}
\end{equation}
L'indeterminazione sulla misura di posizione $\Delta x$ non potr\`{a} essere inferiore al valore di $l$ trovato, dal momento che il singolo elettrone potrebbe appunto trovarsi ovunque entro il cubo di lato $l$. Quindi:
\begin{equation}\label{delta x}
\Delta x \sim \sqrt[3]{\frac{1}{n_{e}}}
\end{equation}
Siamo pronti per il quarto passaggio.
\item
Quarto passaggio.
Supponiamo di conoscere l'indeterminazione $\Delta x$ su una misura di posizione per una certa particella. Per il momento non ci interessa sapere come abbiamo fatto a conoscerla, ci interessa solamente che siamo a conoscenza di quanto vale $\Delta x$. Allora l'indeterminazione con cui può essere  simultaneamente misurata la quantit\`{a} di moto di quella stessa particella non potr\`{a} essere inferiore a:
\begin{equation}\label{delta p}
\Delta p = \frac{\hbar / 2}{\Delta x}
\end{equation}
Ciò discende direttamente dalla relazione di indeterminazione (\ref{Heisenberg}). Se conosco $\Delta x$, $\Delta p$ dovr\`{a} soddisfare la relazione di indeterminazione e pertanto non potr\`{a} essere inferiore al valore dato dalla (\ref{delta p}).
\item
Ora, ulteriore passaggio, nel nostro caso, $\Delta x$ è effettivamente noto, essendo il lato del volumetto (cubico) che contiene un solo elettrone libero, come trovato nella (\ref{delta x}). Allora, tenendo conto della (\ref{delta p}), si ha che $\Delta p$ \emph{non può essere inferiore} a:
\begin{equation}\label{igloo}
\Delta p = \frac{\hbar /2}{\Delta x} \sim \frac{\hbar}{2} \cdot \sqrt[3]{n_{e}}
\end{equation}
Ne consegue che un elettrone libero non possieder\`{a} mai una quantit\`{a} di moto esattamente nulla; ciò è una conseguenza del principio di indeterminazione di Heisenberg, e quindi è una conseguenza deducibile solo tenendo conto della meccanica quantistica. 
\par
Questa quantit\`{a} di moto che l'elettrone possiede genera una pressione; è appunto questa la cosiddetta \emph{pressione degli elettroni degeneri}.
\item
Fin qui tutto bene, passaggi chiari. Ma ora serve una formula che ci dica la pressione esercitata da particelle per le quali è nota la quantit\`{a} di moto. Ecco l'idea: si ricorre  alla formula classica:
\begin{equation}\label{muno}
P=\frac{1}{3} \frac{m \langle v^{2} \rangle N}{V}
\end{equation}
dove $P$ è la pressione esercitata da $N$ particelle di un gas  aventi ciascuna massa $m$ e in moto con velocit\`{a} quadratica media in un volume $V$.
\footnote{La formula, deducibile applicando allo studio dei gas ideali metodi \emph{classici} di indagine statistica, è ampiamente trattata in vari manuali di fisica e chimica liceale. Per la sua deduzione si rimanda a \Cite{caf}. Il lettore è caldamente invitato a leggere l'elegante discussione delle profonde implicazioni di questa legge in \Cite{mun}, pp. 374-380}
\par
Non c'è nulla che ci assicuri \emph{a priori} della validit\`{a} dell'equazione in un contesto in cui gli effetti quantistici sono di rilevante importanza e non possono essere trascurati. Il lettore non tarder\`{a} a notare che sino ad ora si è voluto molto insistere sul fatto che la pressione degli elettroni degeneri è un fatto di natura esclusivamente \emph{quantistica}, non ha senso parlare di pressione degli elettroni degeneri in fisica classica; eppure, adesso, al passaggio finale per il calcolo della pressione, si ricorre ad una formula \emph{classica}. A priori il passaggio è tutt'altro che legittimo, e pertanto in un certo senso può ritenersi giustamente erroneo; vedremo tuttavia che esso conduce a risultati in buon accordo con quelli deducibili in maniera rigorosa con la meccanica quantistica, e per questo motivo è accettabile a posteriori. L'autore si scusa per l'eccessiva pedanteria sulla questione, ma ha ritenuto opportuno insistere sull'argomento al fine di evitare taluni fraintendimenti. Non vi è nulla di difficile nel calcolo della pressione degli elettroni degeneri una volta che sia accettata la validit\`{a} dell'equazione (\ref{muno}) nel contesto in esame. Ma accettarla è tutt'altro che ovvio! \`E solo \emph{per magia} che la meccanica quantistica fornisce un risultato in accordo con quello che verr\`{a} ricavato da questa deduzione!
\item
Nel nostro caso il numero di elettroni liberi  $N=N_{e}$ contenuti nel volume $V$ è dato da $n_{e}=N_{e}/V$, per cui la (\ref{muno}) diventa:
\begin{equation}
P=\frac{1}{3} n_{e} m_{e} \langle v^{2} \rangle
\end{equation}
dove $m_{e}$ è la massa dell'elettrone. Si suppone che $m_{e}$ sia indipendente dallo stato di moto dell'elettrone. Questo perch\'{e}, in generale, si suppone che la velocit\`{a} degli elettroni sia tale da rendere trascurabili eventuali correzioni relativistiche. In altre parole il calcolo qui proposto vale nel caso \emph{non relativistico}: questo sar\`{a} un punto di importanza fondamentale nel seguito.
\par
Ricordando la definizione (non relativistica) di quantit\`{a} di moto $p \equiv m \cdot v$, e ipotizzando che il moto degli elettroni liberi sia dovuto all'indeterminazione sulla quantit\`{a} di moto $\Delta p$ si ha infine:
\begin{equation}\label{finalmente}
P=\frac{1}{3}n_{e}\frac{\Delta p^{2}}{m_{e}}
\end{equation}
\item
Non rimane, ultimo passaggio, che sostituire $\Delta p$ in base alla relazione (\ref{delta p}):
\begin{equation}\label{forma ampia}
P \sim \frac{1}{3} n_{e} \cdot \underbrace{\left( \frac{\hbar^{2}}{4} \cdot \sqrt[3]{n_{e}^{2}} \right)}_{\Delta p^{2}} \cdot \frac{1}{m_{e}}
\end{equation}
Questa è l'espressione per il calcolo della pressione degli elettroni degeneri. Possiamo scrivere questo risultato in forma più raccolta, ma solo per comodit\`{a} di visualizzazione.
La formula definitiva è, dopo qualche passaggio algebrico dalla (\ref{forma ampia}) e sostituendo $\hbar=h/2\pi$:
\begin{equation}\label{stato degenere}
P=a \cdot \frac{h^{2} \ n_{e}\,^{\frac{5}{3}}}{m_{e}}
\end{equation}
dove $a$ è un coefficiente numerico opportunamente scelto.
\end{enumerate}
\subsection{Caratteristiche della pressione di degenerazione}\label{propdeg}
Tre sono le caratteristiche della pressione di degenerazione di cui occorre fare menzione:
\begin{enumerate}
\item[$\bullet$]
Dipendenza dall'inverso della massa delle particelle;
\item[$\bullet$]
Indipendenza dalla temperatura;
\item[$\bullet$]
Dipendenza dalla densit\`{a} di particelle libere.
\end{enumerate}
Tutte queste propriet\`{a} si possono ricavare direttamente dalla (\ref{stato degenere}).
\par
La dipendenza dall'inverso della massa discende dall'essere $m_{e}$ al denominatore nella (\ref{stato degenere}). Questo fatto ci dice che il maggior contributo alla pressione di degenerazione viene proprio dagli elettroni, come preannunciato, piuttosto che dai nuclei.
Per i nuclei il valore della massa è maggiore, e siccome nella (\ref{stato degenere}) la massa compare al denominatore, il termine di pressione associato ai nuclei sar\`{a} minore che nel caso degli elettroni.
\par

\smallskip

La non dipendenza dalla temperatura (cioè il fatto che nella (\ref{stato degenere}) non compare la variabile temperatura $T$) ha conseguenze anche più interessanti.
Una volta che una stella abbia raggiunto la condizione di degenerazione, il solito meccanismo contrazione--riscaldamento imposto dal teorema del viriale viene rotto: per quanto la stella si raffreddi irradiando energia nello spazio, la pressione degli elettroni degeneri si mantiene sempre la stessa. \`E per questo che una nana bianca pur raffreddandosi si mantiene in equilibrio idrostatico. Se, come invece avviene per la pressione termica di un gas perfetto, la pressione di degenerazione dipendesse dalla temperatura, si potrebbe anche dare il caso in cui la nana bianca, dopo una fase di raffreddamento, riprende a contrarsi perch\'{e} la pressione di degenerazione, diminuita con la temperatura, non è più sufficiente a impedirne il collasso. Ciò non è possibile perch\'{e} la pressione di degenerazione è indipendente dalla temperatura.
Per usare un paragone di Margherita Hack al riguardo: ``Un gas degenerato si comporta come un solido, nel senso che la pressione dipende poco o punto dalla temperatura, ma soltanto dalla densit\`{a}. \`E come se noi ci appoggiassimo con tutta la nostra forza a un tavolo di legno o di ferro o di pietra. Che sia caldo o freddo il tavolo supporta ugualmente il nostro peso.'' \citep{hack}.
\par

\smallskip

%La dipendenza dalla densit\`{a} volumetrica del numero di elettroni (nella (\ref{stato degenere}) compare al numeratore il termine $n_{e}^{5/3}$ da informazioni sulle rapporto tra massa e raggio di una nana bianca. Si torner\`{a} nel seguito su questo punto.
\`E bene sottolineare ancora una volta che tutte queste propriet\`{a} valgono nell'approssimazione non relativistica. In seguito si vedr\`{a} le importanti conseguenze di lavorare in regime relativistico, e questo ci condurr\`{a} direttamente al famosissimo limite di Chandrasekhar.
\subsection{Quando bisogna tener conto della pressione di degenerazione?}
Perch\'{E}, quando si è parlato delle stelle di sequenza principale, si è parlato di pressione del plasma stellare e di pressione di radiazione, ma non si è mai fatto alcun riferimento alla pressione degli elettroni degeneri?
\par
La condizione di degenerazione diventa importante quando i valori di densit\`{a} del gas diventano sufficientemente grandi.
La densit\`{a} minima alla quale, per un certo gas, bisogna tener conto anche della pressione degli elettroni degeneri è detta \emph{densit\`{a} degli stati quantici}. Se indichiamo questa densit\`{a} con $\rho_{q}$, questa è data dalla relazione:
\begin{equation}
\rho_{q}= \left( \frac{2\pi m_{e} k T}{h^{2}} \right)^{\frac{3}{2}}
\end{equation}
con $h$ costante di Planck, $k$ costante di Boltzmann (introdotta alla sezione \ref{aumentot}) $m_{e}$ massa dell'elettrone e $T$ temperatura assoluta.
Questa relazione si ricava imponendo che il momento della (\ref{igloo}) sia eguale al momento termico medio degli elettroni.
\par
\`E importante notare come questa densit\`{a} $\rho_{q}$ dipenda dalla temperatura $T$: più freddo è il gas più bassa è la densit\`{a} a cui la pressione di degenerazione diventa non trascurabile. Questo ci dice che i gas degeneri sono gas \emph{freddi}. La condizione di degenerazione si ha a densit\`{a} tanto più basse quanto più bassa è la temperatura.
\par
Per il Sole \emph{mediamente} il rapporto tra densit\`{a} volumetrica di elettroni liberi ($n_{e}$, gi\`{a} introdotta precedentemente) e densit\`{a} degli stati quantici $\rho_{q}$ vale, come si potrebbe verificare con un semplice calcolo, circa:
\begin{displaymath}
\rho_{q} \sim 55 n_{e}
\end{displaymath}
cioè mediamente nel Sole gli effetti dovuti alla degenerazione sono trascurabili.
\section{Giganti rosse}\label{giganti rosse}
\subsection{Nucleosintesi del carbonio}
Stelle di massa superiore a $0.5M_{\odot}$, nel processo di contrazione gravitazionale del nocciolo, raggiungono la temperatura necessaria affinché si abbia l'innesco delle reazioni di fusione nucleare dell'elio $^{4}He$, che si è gi\`{a} visto essere dell'ordine dei 100 milioni di kelvin.
Per quale ragione il ``bruciamento'' dell'elio richiede temperature un ordine di grandezza maggiori rispetto alle temperature richieste per la fusione dell'idrogeno? La ragione è molto semplice: se nel caso dell'idrogeno la repulsione coulumbiana da vincere per far avvicinare due nuclei era quella dovuta alla repulsione tra 2 protoni, nel caso dell'elio la repulsione coulumbiana da vincere è invece quella, ben maggiore della prima, tra 4 protoni (due di un nuclo di elio e due dell'altro nucleo di elio).
\par
Analizziamo più in dettaglio cosa avviene. Esaurito o quasi l'idrogeno del core, venute meno le reazioni di fusione dell'idrogeno, il nocciolo di una stella inizia a contrarsi riscaldandosi. Si è gi\`{a} visto che se la massa $M$ della stella è troppo piccola ($M<0.5 M_{\odot}$), causa l'insorgere di fenomeni quantistici, la contrazione del nocciolo ad un certo punto si arresta e la stella si stabilizza nello stadio di nana bianca, ove rester\`{a} fino al sopraggiungere della morte termica. Ma se invece la massa della stella è un po' più grande ($M>0.5M_{\odot}$, ed è questo il caso che si considera in questa sezione) allora il nocciolo ha modo di raggiungere la temperatura di 100 milioni di kelvin. A questa temperatura, i nuclei di elio presenti in abbondanza nel nocciolo\footnote{Prodotti dalle reazioni di fusione dell'idrogeno quando la stella era ancora in sequenza principale} possiedono un'energia cinetica sufficiente per vincere la repulsione elettrostatica reciproca e portarsi vicini gli uni agli altri. La fusione di due nuclei di elio $^{4}He$ origina un nucleo di berillio $^{8}Be$. Ma c'è un problema: il nucleo di berillio non è stabile, e tende a scindersi in due nuclei di elio nel tempo di $10^{-16}$ secondi! Tuttavia, per quanto breve sia la sua vita, il nucleo di Berillio cos\`{\i}{} originatosi ha tutto il tempo, data la frequenza degli urti, di interagire con un altro nucleo di elio $^{4}He$ formando con esso un nucleo di carbonio $^{12}C$, che è stabile. Si ha la nucleosintesi del carbonio a partire da tre nuclei di elio, attraverso un urto triplo che avviene in due fasi comunque separate nel tempo (il terzo nucleo urta il nucleo di Berillio).
\footnote{In misura minore, nel regime di temperatura della fusione dell'elio anche altre reazioni di nucleosintesi si rendono possibili.}
I nuclei di elio sono anche chiamati particelle $\alpha$: per questo la reazione di fusione ora descritta va sotto il nome di processo tripla--$\alpha$. La nucleosintesi del carbonio tramite procedimento tripla--$\alpha$ è una reazione di fusione nucleare che libera energia. Stelle di massa $M>0.5 M_{\odot}$, una volta raggiunte nel nocciolo le temperature sufficienti, entrano in una nuova fase di stabilit\`{a}, questa volta però non più grazie all'energia prodotta dalla combustione dell'idrogeno (ormai praticamente quasi assente nel nocciolo), ma grazie alla fusione dell'elio. Questa fase di stabilit\`{a} è ancora una volta contraddistinta, come la precedente in sequenza principale, da equilibrio termico ($\Delta E=0$, cioè la perdita di energia per irraggiamento viene compensata perfettamente dalla produzione di energia nel nocciolo) ed equilibrio idrostatico.
\subsection{Espansione dell'inviluppo esterno}
L'innesco delle reazioni di fusione dell'elio è accompagnato da una espansione degli strati esterni della stella.
In molti testi questo fatto è riferito senza troppe spiegazioni e, il più delle volte, argomentato sulla base di motivazioni certamente intuitive ma purtroppo scarsamente rigorose e soddisfacenti. La spiegazione più corretta di questo apparentemente strano comportamento degli strati esterni si basa sul teorema del viriale. Il lettore non dovrebbe stupirsi più di questo fatto, visto che buona parte del comportamento di una stella può essere dedotto facendo ricorso al teorema del viriale o a sue generalizzazioni. La deduzione purtroppo in questo caso è ardua: sebbene concettualmente semplice, dal punto di vista matematico essa richiede l'impiego del calcolo differenziale. \`E solo questa ragione che spinge a non presentare in questa sede una simile deduzione. Il lettore tenga comunque presente che il fatto che la stella, innescate le reazioni di fusione dell'elio, vada incontro ad una espansione degli strati esterni è una conseguenza perfettamente deducibile a mezzo del teorema del viriale.
\par
A seguito dell'espansione degli strati esterni, si assiste ad una diminuzione della temperatura \emph{superficiale} della stella. Anche questo si può spiegare facilmente con il teorema del viriale. Se la contrazione produce un incremento nei valori di temperatura, l'espansione ne determina invece una diminuzione. Per la legge dello spostamento di Wien, la stella apparir\`{a} rossa. Per questo essa assume la denominazione di \emph{gigante rossa}. Nel diagramma HR si colloca in alto a destra.
\subsection{Tempo di permanenza nello stadio di gigante rossa}
Il tempo di permanenza di una stella nella fase di gigante rossa è comunque inferiore a quello di permanenza della stessa in sequenza principale, e questo  per due ragioni:
\begin{itemize}
\item
La quantit\`{a} di elio nel nocciolo è inferiore rispetto alla quantit\`{a} di idrogeno di cui una stella in sequenza principale mediamente può disporre;
\item
Il difetto in massa nella fusione dell'elio è molto inferiore rispetto a quello delle reazioni nucleari che avvengono in una stella di sequenza principale e pertanto, a parit\`{a} di energia rilasciata, viene consumata una quantit\`{a} maggiore di elio.
\end{itemize}
In termini più immediati, una gigante rossa ha meno combustibile e nel contempo è costretta a consumarne ad un ritmo maggiore.
\par
Quando anche l'elio nel nocciolo inizier\`{a} ad esaurirsi, la stella uscir\`{a} nuovamente dalla fase di stabilit\`{a} ed il suo nocciolo proceder\`{a} verso una successiva fase di contrazione gravitazionale, con conseguente aumento di temperatura (il tutto in accordo con il teorema del viriale).
Il futuro della stella ancora una volta è segnato dalla sua massa.
\subsection{Fasi terminali di sviluppo di giganti rosse di massa $M<8 M_{\odot}$}
Giganti rosse di massa \emph{complessiva} inferiore a $8 M_{\odot}$, nella fase di contrazione del nocciolo, che fa seguito all'esaurisi delle riserve di elio, \emph{non} raggiungono temperature sufficienti ad innescare ulteriori reazioni di nucleosintesi.
\par
Vediamo più in dettaglio quello che succede.
Via via che le scorte di elio nel nocciolo si fanno sempre più carenti e il nocciolo stesso prosegue nella sua fase di contrazione, nelle regioni più esterne le temperature diventano tali per cui si verificano reazioni di fusione dell'idrogeno.
Negli strati esterni l'idrogeno è infatti ancora presente, non è stato consumato durante la permanenza in sequenza principale perch\'{e}, come il lettore certo ricorder\`{a}, in genere non si ha rimescolamento tra il materiale del core e quello dell'involucro esterno. A questo punto però anche la temperatura delle regioni esterne (non superficiali comunque) è sufficientemente elevata da consentire alle reazioni di nucleosintesi dell'elio di avere luogo. In uno strato immediatamente attorno al nocciolo centrale inoltre cominciano ad innescarsi le reazioni di fusione dell'elio (il tripla--$\alpha$). Nel nocciolo, dove ormai quasi tutto l'elio è stato convertito in carbonio, la temperatura non è però (e, come si vedr\`{a} tra breve, non potr\`{a}, per queste stelle, mai essere) abbastanza grande perch\'{e} prendano avvio le reazioni di fusione del carbonio.
\par
La stella è una sorta di ``cipolla'' (immagine questa dipinta da numerosi testi divulgativi) con al centro un nocciolo inerte, costituito per buona parte da carbonio, uno strato più esterno dove ancora avvengono le reazioni di fusione dell'elio, uno strato ancora più esterno dove è attiva la fusione nucleare dell'idrogeno ed infine uno strato superficiale più freddo, composto da idrogeno ed elio inerti.
\par
Ad un certo punto, nel suo processo di contrazione, il nocciolo diventa degenere. La pressione degli elettroni degeneri, non più trascurabile, finisce infine per impedire ogni ulteriore contrazione delle regioni centrali della stella. Per stelle di massa complessiva inferiore a $8 M_{\odot}$, il collasso del nucleo centrale viene arrestato prima che questo abbia raggiunto temperature sufficienti per la fusione del carbonio.
Il nocciolo centrale andr\`{a} cos\`{\i}{} a stabilizzarsi nello stadio di \emph{nana bianca}, sostenuto contro un ulteriore collasso dalla pressione di degenerazione degli elettroni, e quindi stabile in una condizione di equilibrio idrostatico, ma non termico.
In altre parole lo stadio di nana bianca, che abbiamo gi\`{a} incontrato come tappa conclusiva dello sviluppo di stelle di massa inferiore a $0.5 M_{\odot}$ rappresenta anche lo stadio evolutivo finale di giganti rosse poco massicce ($M<8M_{\odot}$). Valgono tutte le considerazioni gi\`{a} espresse sulle nane bianche, e cioè il procedere inesorabile verso la morte termica, il fatto che comunque il tempo necessario perch\'{e} una nana bianca si spenga definitivamente in una nana nera è estremamente lungo ecc.
\par
Questo è ciò che succede al nocciolo. Cosa succede invece alle regioni esterne della stella?
A seguito di ``convulsioni profonde, combinate con la pressione della radiazione e altre forze'' \Cite{balick}, gli strati via via più superficiali della stella vengono scagliati nello spazio. Nella fasi iniziali la velocit\`{a} di espulsione è compresa tra 10 e 20 kilometri al secondo, in seguito diventa anche maggiore \Cite{balick}.
Si va formando una \emph{nebulosa planetaria}.
Sono oltre 2000 le nebulose planetarie oggi conosciute. Osservate al telescopio, le nebulose planetarie si rivelano essere oggetti dalle forme particolarmente curiose. Modelli teorici che spieghino in qualche misura come l'espulsione di gas nelle fasi terminali di sviluppo di una gigante rossa possa originare oggetti dalle forme tanto inusuali e diverse tra loro sono ancora oggi oggetto di elaborazione. Secondo i modelli più recenti pare plausibile che le nebulose planetarie si originino per espulsione dei gas in fasi separate nel tempo; pare inoltre che un ruolo importante venga svolto dai campi magnetici. L'argomento, estremamente interessante, non sar\`{a} qui ulteriormente approfondito. Per approfondire si segnala \Cite{balick}.
\subsection{Fasi terminali di sviluppo di giganti rosse di massa $M>8 M_{\odot}$}
Diversamente dal caso precedente, stelle di massa maggiore a $8M_{\odot}$ nella fase di collasso del nocciolo centrale, che segue all'esaurimento dell'elio, raggiungono temperature al centro sufficienti all'attivazione delle reazioni di fusione del carbonio.
Ma c'è di meglio: una volta che anche le quantit\`{a} di carbonio disponibili nel nocciolo diventano presto insufficienti, il nocciolo va soggetto nuovamente a contrazione e corrispondente riscaldamento; e cos\`{\i}{} facendo raggiunge quella temperatura indispensabile all'innesco di altre reazioni nucleari che coinvolgono ossigeno, neon e magnesio. Stelle di massa maggiore a $8M_{\odot}$ non solo raggiungono la temperatura richiesta per la fusione del carbonio (cosa questa che gi\`{a} non avviene per le giganti rosse meno massicce) ma proseguono anche oltre a successive reazioni di fusione. Ogni volta che un combustibile si esaurisce, il nocciolo si contrae e si riscalda, \emph{senza mai raggiungere lo stato degenere}, e nel nocciolo si sviluppano via via che la temperatura aumenta nuove reazione di fusione, fino alla fusione del ferro.
\par
Ad un certo punto la nostra gigante rossa massiccia avr\`{a} una struttura a ``cipolla'', un po' come si è descritto per giganti rosse meno massicce. In questo caso però la struttura a cipolla risulta molto più articolata: allo strato superficiale freddo ed inerte, fa seguito, procedendo verso l'interno, un primo strato in cui si verifica la fusione dell'idrogeno; un secondo strato in cui si verifica la fusione dell'elio; un terzo dove si verifica la fusione del carbonio; e cos\`{\i}{} via fino al nocciolo centrale, dove supponiamo sia in corso la fusione del silicio  e dello zolfo a formare ferro.
\par
Una volta che le quantit\`{a} di silicio e zolfo cominciano a essere insufficienti al solito il nocciolo inizia a contrarsi, aumentando la temperatura. Il collasso non è ostacolato dalla pressione di degenerazione.
\footnote{La pressione di degenerazione non raggiunge (ne raggiunger\`{a}) mai un valore sufficiente per sostenere il collasso. Questo perch\'{e} il nucleo è troppo massiccio. Si vedr\`{a} in sezione \ref{chandra} che, per masse maggiori a $1.4M_{\odot}$ il collasso \emph{non} può essere arrestato dalla pressione di degenerazione.}
%Questo perch\'{e} la densit\`{a} degli stati quantici dipende dalla temperatura, qui estremamente elevata: ne consegue che, in questi regimi di temperatura, gli effetti legati alla degenerazione possono essere trascurati. 
Si arriva ad un punto in cui la temperatura del nocciolo risulta pari a 10 miliardi di gradi. A questa temperatura si può verificare la fusione del ferro.
\par
Ma proprio a questo punto succede qualcosa.
Un nucleo di ferro $^{56}Fe$ si trasforma in 13 nuclei di elio e quattro neutroni. Diversamente da quanto avviene per tutte le altre reazioni nucleari trattate inprecedenza, la reazione ferro--elio è una reazione endoergonica: richiede energia invece che produrla.
L'energia richiesta dalla reazione ferro--elio va a scapito dell'energia termica del nucleo, che bruscamente si raffredda.
La stella entra in una fase di instabilit\`{a} a seguito del repentino raffreddamento del core, e i suoi strati esterni vanno infine incontro a violenta  esplosione. \`E il fenomeno della \emph{supernova}
\subsection{Supernavae}
Giganti rosse di massa $M>8 M_{\odot}$ terminano la loro vita in modo esplosivo, dando origine al fenomeno della \emph{supernova}. Un modello preciso dei meccanismi di instabilit\`{a} che sono alla base del fenomeno della supernova non sar\`{a} presentato in questa sede. Si intende nel seguito limitarsi a riportare alcune informazioni di carattere generale sulle supernovae.
\par
Si premette sin d'ora che le supernovae di cui si è fatto e si far\`{a} menzione in seguito, cioè quelle supernovae intese come esplosione di giganti rosse massicce, sono più propriamente dette \emph{supernovae di tipo II}. Le supernovae di tipo I si originano in sistemi binari costituiti da una nana bianca di massa prossima a 1,4 masse solari\footnote{Il perch\'{e} di questo valore sar\`{a} meglio chiarito nella sezione \ref{chandra}.} e da una stella prossima allo stadio di gigante rossa. L'esplosione \`e scatenata da un trasferimento di materia dalla compagna alla nana bianca \Cite{freedman}. Le \mbox{supernovae I} presentano molteplici elementi di diversit\`{a} rispetto alle \mbox{supernovae II:} raggiungono nel punto di massimo luminosit\`{a} maggiori, hanno un differente andamento della curva di luminosit\`{a} tipica e sono utilizzate (quelle di tipo I, e solo quelle) come candele standard di riferimento nella misurazione delle distanze extragalattiche\footnote{Le supernovae Ia raggiungono tutte la medesima luminosit\`a assoluta massima \citep{burnham, hack}.} Non ci occuperemo nel seguito di supernovae di tipo I.
\par
Nel momento in cui una gigante rossa esplode come supernova II, gli strati esterni della stella vengono proiettati nello spazio con velocit\`{a} di decine di migliaia di kilometri al secondo \Cite{rosino}.
Nel contempo la stella aumenta la propria luminosit\`{a} al punto da brillare come miliardi di stelle normali \Cite{rosino}; alle volte una supernova può raggiungere luminosit\`{a} a tal punto elevate da risultare comparabili con quelle dell'intera galassia che la ospita \Cite{battistini}.
\par
La curva di luminosit\`{a} caratteristica per una supernova di tipo II presenta un aumento non brusco nel periodo precedente al massimo; quest'ultimo è seguito in un primo tempo da una diminuzione lenta della luminosit\`{a} dell'astro e solo un centinaio di giorni dopo l'avvenuto massimo la decrescita della luminosit\`{a} diventa rapida \Cite{burn}.
\par 
L'inviluppo espulso dalla stella andr\`{a} a formare una massa di gas in espansione ad altissima velocit\`{a} (\emph{nebulosa residuale}).
Tra le nebulose residuali più note, degna di menzione è certamente la \emph{Crab Nebulsa}, nella costellazione del Toro. Questa nebulosa residuale è associata all'esplosione di una supernova avvenuta, all'interno della nostra galassia, nel 1054. In data attuale le misurazioni più recenti permettono una stima della velocit\`{a} di espansione dei gas di questa nube pari a circa 1000 km/sec \citep{hack}.
Cosa succeda al nocciolo centrale della gigante rossa ad esplosione avvenuta sar\`{a} discusso ampiamente alla sezione \ref{chandra}.
\par
Le esplosioni di supernovae sono eventi molto rari; si stima che in una galassia si abbia una supernova ogni circa tre secoli \citep{burnham}. Le supernovae esplose nella nostra galassia di cui si abbia notizia negli ultimi secoli sono appena quattro: la grande supernova esplosa nella costellazione del Lupo nel 1006, di cui sono rimaste diverse citazioni e la cronaca dell'astrologo islamico Ali Ibn Ridwan; la supernova esplosa nel Toro il 4 luglio del 1054, di cui ci sono rimaste cronache cinesi e la cui nebulosa residuale è ancora oggi osservabile (la famosa \emph{Crab Nebulsa}); la supernova in Cassiopea del 1572, di cui l'astronomo Tycho Brahe ci ha lasciato una accurata cronaca; ed infine la supernova esplosa nel 1604 nell'Ophiuco di cui Keplero, Galileo ed altri ci hanno lasciato cronache.
\footnote{L'argomento delle supernovae storiche , di sicuro interesse, non sar\`{a} per brevit\`{a} ulteriormente approfondito. Riferimenti per ulteriori letture saranno dati in bibliografia.}
\`E interessante notare come, data l'elevata luminosit\`{a} massima che caratterizza le supernovae e il fatto che si tratti di supernovae esplose nella nostra stessa galassia, ciascuno degli eventi storici ora menzionati sia stato osservato a occhio nudo, senza l'ausilio dei mezzi di indagine dell'astronomia telescopica, mezzi peraltro a quei tempi non ancora in uso.
\par
Dato comunque il numero elevatissimo di galassie che vengono oggi osservate dagli astronomi, si scopre mediamente più di una supernova extragalattica a settimana \Cite{battistini}.
\par
Se si considera che il numero di stelle aventi massa maggiore di $8M_{\odot}$ è piuttosto elevato, viene da chiedersi per quale motivo siano cos\`{\i}{} rari gli eventi di esplosioni di supernovae. La risposta è da ricercarsi nel fatto che, durante le fasi precedenti all'esplosione, la stella, sin dall'epoca della sequenza principale, ha in atto diversi meccanismi con cui disperde nello spazio parte della sua massa. Questi meccanismi sono almeno di tre tipi \Cite{rosino}:
\begin{itemize}
\item
Una stella può perdere massa in seguito ad un'elevata rotazione: per effetto delle intense forze centrifughe che si sviluppano, una stella in rapida rotazione può emettere materia dal suo equatore, materia che si andr\`{a} a depositare sul piano equatoriale formando anelli concentrici in espansione. \`E questo il caso di stelle di classe $Of$ o $Be$ o anche delle stelle Wolf-Rayet;
\item
In stelle molto massicce, si può avere emissione di massa per effetto di un'intensa pressione di radiazione;
\item
In stelle di altissima temperatura, come stelle di classe $O$, si può verificare inoltre che le particelle che compongono gli strati più esterni, per effetto del moto di agitazione termica, si muovano a velocit\`{a} superiori alla velocit\`{a} di fuga in quella regione.
\end{itemize}
Da  segnalare l'esplosione, avvenuta nell'ormai lontano 23 febbraio del 1987, ore 7.35 locali, di una supernova nella Grande Nube di Magellano, una galassia vicina alla nostra via Lattea. Si è trattato della prima supernova extragalattica visibile a occhio nudo. L'esplosione di 1987A, nome con il quale la supernova è stata poi designata, è stata importante per almeno due ragioni: di essa è stato possibile seguire in diretta le diverse fasi dell'esplosione ed individuare anche la stella progenitrice; è stato inoltre osservato, da vari rilevatori, il flusso di neutrini emesso nell'esplosione.
Una documentazione fotografica delle varie fasi dell'evento è riportata in \Cite{caf}, p. F322.
\par

\smallskip

Nel seguito dovremo tornare a parlare di supernovae. Rimane infatti ancora da chiarire cosa rimanga, ad esplosione avvenuta, del nocciolo centrale. Ci si potrebbe aspettare che questo collassi gravitazionalmente fino a quando il subentrare della pressione degli elettroni degeneri non gli permette di stabilizzarsi nello stadio di nana bianca. In effetti se nell'esplosione il nocciolo riesce a portare la propria massa al di sotto di un limite critico, pari a $1.4 M_{\odot}$, questo è proprio quello che avviene: ancora una volta la stella termina la propria vita nello stadio di nana bianca, fino al sopraggiungere della morte termica. Generalmente la massa residua è però superiore a questo limite. In tal caso, per ragioni che vedremo in sezione \ref{chandra}, la stella \emph{non può stabilizzarsi nello stadio di nana bianca}. Si vedr\`{a} che la stella terminer\`{a} allora la propria vita nello stadio di pulsar, oppure di stella di quark e gluoni o ancora di buco nero a seconda del valore della massa residua. Ma prima di descrivere questi ultimi casi, riassumiamo in breve quali sono i tratti distintivi e qualificanti delle nane bianche.
\section{Caratteri principali delle nane bianche}\label{nane bianche}
Vengono presentati in forma riassuntiva i principali elementi caratterizzanti delle nane bianche. L'esposizione si ispira a \Cite{burn}:
\begin{enumerate}
\item
\emph{Diametro}:
le nane bianche sono caratterizzate da valori del diametro molto ridotti. Le dimensioni di una nana bianca risultano comparabili con le dimensioni di pianeti di tipo terrestre. Sirio B ha un diametro stimato non superiore a 19000 miglia\footnote{Burnham riporta i valori di lunghezza del diametro espressi in unit\`{a} del sistema anglosassone. Vale il fattore di conversione 1mi = 1,6Km.}. La stella di Van Maanen si stima abbia un diametro di 7800 miglia, cioè è più piccola della Terra. Sono note nane bianche di diametro anche inferiore: LP 357--186, scoperta da W. J. Luyten nel 1962 nella costellazione del Toro, pare avere un diametro calcolato pari a 1200 miglia e LP 768--500, la cui scoperta, nella costellazione della Balena, sempre ad opera di Luyten, è stata annunciata nel novembre del 1963, ha un diametro calcolato pari a 900 miglia.
\item
\emph{Massa}:
stime attendibili della massa delle nane bianche può essere fatta solo per quelle nane bianche appartenenti a sistemi binari.
Solo in questo caso risulta possibile, impiegando l'analogo della terza legge di keplero al sistema binario di stelle, risalire al valore della massa totale delle due due componenti partendo da misure del periodo di rotazione e della distanza media tra le due stelle. \footnote{Questo è il metodo che si usa in astronomia per misurare la massa delle stelle, non solo delle nane bianche. In alcuni casi particolari è anche possibile risalire dal valore della massa totale delle due componenti alla massa di ogni singola componente.}Ne sono un esempio Sirio B, 40 Eridani e Procione B, i cui valori di massa calcolati risultano rispettivamente pari a 0.98, 0.44 e circa 0.65 masse solari. In generale le nane bianche hanno massa abbastanza prossima al valore della massa attuale del Sole. Per ragioni che saranno descritte in sezione \ref{chandra} non è possibile che esistano nane bianche di massa superiore a circa $1.4 M_{\odot}$.
\item
\emph{Densit\`{a}}:
i valori di densit\`{a} per una nana bianca sono molto elevati (anche se nel seguito incontreremo oggetti astronomici di densit\`{a} anche molto maggiore). In una bianca la massa del Sole viene ad essere infatti concentrata in un volume molto più piccolo.
La densit\`{a} di una nana bianca può raggiungere il valore di una tonnellata per centimetro cubo ($10^{9}$ kg/m$^{3}$).
\item
\emph{Luminosit\`{a}}:
le nane bianche non sono oggetti molto luminosi. Nel diagramma HR occupano la porzione in basso a sinistra, corrispondente appunto a valori relativamente modesti di luminosit\`{a}. La ragione di questo fatto è essenzilmente da ricercarsi, come gi\`{a} ampiamente discusso, nella ridotta superficie emettente che possiedono (vedi legge di Stefan--Boltzmann).
Volendo fornire alcuni esempi di riferimento, basti pensare che HZ 29, una nana bianca nella costellazione dei Cani da Caccia, considerata uno tra i più luminosi oggetti stellari appartenenti a questa categoria, non supera 1/40 della luminosit\`{a} del Sole. Per HZ 29 si calcola una magnitudine assoluta pari a +8.9. Le due nane bianche scoperte da Luyten nei primi anni Sessanta, che si è visto prima essere caratterizzate da valori del diametro estremamente ridotti, hanno magnitudine assoluta: LP 357--186 pari a +16.5 ed LP 768--500 probabilmente inferiore a +17.
\item 
\emph{Temperature}:
oltre met\`{a} delle nane bianche oggi conosciute appartengono alla classe spettrale A, sono caratterizzate cioè da una temperatura superficiale tra gli 8000 e i 10000 kelvin. Nane bianche di classe spettrale F sono piuttosto rare. esistono comunque alcuni (pochi) esempi di nane bianche con temperatura superficiale anche inferiore: la stella di Van Maanen è di classe G e W489 è classificata di tipo K.
\end{enumerate}


