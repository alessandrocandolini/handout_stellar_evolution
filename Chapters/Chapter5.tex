
%*******************************************************
% Chapter 5
%*******************************************************
\chapter{Oggetti estremi}\label{buchi neri}
\minitoc\mtcskip
Nel caso di masse in gioco maggiori di 3--4 masse solari, fino a pochi anni fa si riteneva che ormai nulla potesse impedire il collasso gravitazionale del nocciolo residuo dell'esplosione della supernova; la contrazione sarebbe proseguita fino a che tutta la massa sarebbe rimasta concentrata in un punto di dimensione nulla, originando quello che il fisico americano John A. Wheeler ha chiamato \emph{buco nero}.
Oggi si è propensi a considerare l'ipotesi che, tra la stella di neutroni e il buco nero esista in realt\`{a} un ulteriore caso intermedio, rappresentato dalle \emph{stelle di plasma di quark e gluoni}.
Questa ipotesi pare confermata dalla recente osservazione, da parte dell'osservatorio orbitante per raggi X Chandra, di due oggetti, RXJ--1856 e 3C--58, di dimensioni pari a poco più di una decina di kilometri \Cite{regge}.
\par
In particolare l'oggetto RXJ--1856, un residuo di supernova distante 400 anni luce da noi, è stato studiato da Jeremy Drake, dell'Harvard Smithsonian Center di New York. I risultati dell'indagine forniscono per l'oggetto un diametro stimato pari ad 11 km circa, una temperatura inferiore al milione di kelvin e una luminosit\`{a} particolarmente elevata nella banda X \Cite{cardone}; tali propriet\`{a} sono interpretabili ammettendo che esso, e lo stesso dicasi per 3C--58, sia costituito da quark liberi originatisi dalla scissione dei neutroni durante la fase di collasso gravitazionale.
Maggiori dettagli sul plasma di quark e gluoni, i tentativi di produrlo in laboratorio e il deconfinamento dei quark (violazione della libert\`{a} asintotica) che questo esotico stato della materia comporta sono reperibili in \Cite{antinori}.
\par
In una ipotetica scala degli oggetti stellari più densi, le stelle di plasma di quark e gluoni si pongono immediatamente dopo le stelle di neutroni, delle quali sarebbero circa 2 o 3 volte più dense \Cite{cardone}, e i buchi neri.
%\begin{figure}
%\begin{center}
%\includegraphics[width=15cm]{surf1}
%\end{center}
%\caption{Due oggetti forse appartenenti alla categoria delle stelle di plasma di quark e gluoni: RXJ--1856 (a sinistra) e 3C--58 (a destra) \Cite{cardone}.}
%\end{figure}
\setlength{\unitlength}{1mm}
\begin{figure}[!b]
\begin{center}
\begin{picture}(120,190)%(10,20)
%\put(0,0){\framebox(120,170)[cc]{}}
\put(45,160){\framebox(30,08)[cc]{Globuli di Bok}}
\put(60,160){\line(0,-1){15}}
\put(60,150){\line(-1,0){45}}
\put(62,145){\makebox(50,15)[lc]{\shortstack[l]{\footnotesize Contrazione gravitazionale \\ \footnotesize Aumento temperatura}}}
\put(15,152){\makebox(45,06)[cb]{$\scriptstyle M<0.07 M_{\odot}$}}
\put(15,150){\line(0,-1){5}}
\put(03,137){\framebox(24,08)[cc]{Nane brune}}
%(50,16)[lc]{Contrazione gravitazionale \par Aumento temperatura}}
\put(33,141){\makebox(54,5)[cb]{\footnotesize{Innesco fusione $H$}}}
%89=87+2di margine
\put(80,141){\makebox(20,5)[lb]{$T>10^{7} K$ }}
%\put(94,140){\circle{10}}
\put(60,140){\line(0,-1){5}}
\put(30,123){\framebox(60,12)[cc]{\shortstack[c]{Stelle di sequenza principale\\ \small (stabilit\`a)}}}
\put(60,123){\line(0,-1){6}}
\put(46,107){\makebox(28,10)[cc]{\shortstack[c]{\footnotesize L'idrogeno va \\ \footnotesize esaurendosi}}}
\put(60,107){\line(0,-1){13}}
\put(20,099){\vector(0,-1){5}}
\put(60,099){\line(-1,0){40}}
\put(20,101){\makebox(40,06)[cb]{$\scriptstyle M<0.5 M_{\odot}$}}
\put(05,086){\framebox(30,08)[cc]{Nane bianche}}
\put(40,084){\makebox(40,10)[cc]{\shortstack{\footnotesize{Innesco fusione $He$} \\ \footnotesize{(tripla--$\alpha$)}}}}
\put(80,085){\makebox(20,10)[lc]{$T>10^{8} K$ }}
\put(60,084){\line(0,-1){3}}
\put(45,070){\framebox(30,11)[cc]{\shortstack[c]{Giganti rosse\\ \small (stabilit\`a)}}}
\put(60,070){\line(0,-1){3}}
\put(42,062){\makebox(36,05)[cc]{\footnotesize L'elio va esaurendosi}}
\put(60,062){\line(0,-1){4}}
\put(60,058){\line(-1,0){40}}
\put(60,058){\line(1,0){40}}
\put(20,058){\line(0,-1){5}}
\put(100,058){\line(0,-1){5}}
\put(20,59){\makebox(22,5)[cb]{$\scriptstyle M<8M_{\odot}$}}
\put(78,59){\makebox(22,5)[cb]{$\scriptstyle M>8M_{\odot}$}}
\put(5,48){\makebox(30,5)[cc]{\footnotesize No fusione carbonio}}
\put(20,48){\vector(0,-1){5}}
\put(5,35){\framebox(30,8)[cc]{Nane bianche}}
\put(85,48){\makebox(30,5)[cc]{\footnotesize Fusione carbonio}}
\put(100,48){\line(0,-1){3}}
\put(85,40){\makebox(30,5)[cc]{\footnotesize [\ldots]}} 
\put(100,40){\line(0,-1){3}}
\put(85,32){\makebox(30,5)[cc]{\footnotesize Fusione $Fe$}} 
\put(100,32){\line(0,-1){4}}
\put(85,20){\framebox(30,8)[cc]{Supernovae}}
\put(100,20){\line(0,-1){3}}
\put(85,12){\makebox(30,5)[cc]{\footnotesize Strati esterni}} 
\put(100,12){\vector(0,-1){3}}
\put(82,02){\framebox(36,7)[cc]{Nebulosa residuale}}
\put(85,24){\line(-1,0){5}}
\put(66,20){\makebox(14,8)[cc]{\footnotesize nocciolo}}
\put(66,24){\line(-1,0){54}}
\put(12,24){\line(0,-1){3}}
\put(02,16){\makebox(20,5)[cc]{$\scriptstyle M_{n}<1.4 M_{\odot}$}}
\put(12,16){\vector(0,-1){4}}
\put(4,02){\framebox(16,10)[cc]{\shortstack[c]{Nane\\bianche}}}
\put(36,24){\line(0,-1){3}}
\put(22,16){\makebox(28,5)[cc]{$\scriptstyle 1.4M_{\odot}<M_{n}<3 M_{\odot}$}}
\put(36,16){\vector(0,-1){4}}
\put(28,02){\framebox(15,10)[cc]{\shortstack[c]{Stelle di\\neutroni}}}
\put(60,24){\line(0,-1){3}}
\put(50,16){\makebox(20,5)[cc]{$\scriptstyle M_{n}\gg 3 M_{\odot}$}}
\put(60,16){\vector(0,-1){4}}
\put(52,02){\framebox(16,10)[cc]{\shortstack[c]{Buchi\\neri}}}
\put(57,24){\line(0,1){3}}
\put(47,27){\makebox(24,5)[cc]{$\scriptstyle M_{n}\gtrsim 3 M_{\odot}$}}
\put(57,32){\vector(0,1){3}}
\put(43,35){\framebox(28,10)[cc]{\shortstack[c]{Stelle di plasma\\quark e gluoni}}}
\end{picture}
\end{center}
\caption{Schema generale}
\end{figure}
Per masse in gioco molto maggiori, il nocciolo in fase di contrazione non potr\`{a} stabilizzarsi neppure nello stadio di stella di quark. In questo caso, come preannunciato all'inizio della sezione, il collasso gravitazionale è inarrestabile, e prosegue fino a che tutta la massa del corpo stellare risulta compressa in un solo punto. Si forma un buco nero. Un buco nero è una regione chiusa di spazio--tempo dalla quale nessun segnale o informazione può evadere verso un ipotetico osservatore esterno causa l'intensit\`{a} elevatissima del campo gravitazionale presente \Cite{hawking}.
Buchi neri possano formarsi in circostante diverse: buchi neri primordiali possono essersi formati laddove erano casualmente presenti concentrazioni di materiale cosmico particolarmente dense \Cite{hawking}; secondo alcune teorie recenti pare che buchi neri possano formarsi quotidianamente per impatto dei raggi cosmici con le particelle che compongono gli strati alti dell'atmosfera terrestre; inoltre non è da escludere che gi\`{a} con la prossima generazione di acceleratori di particelle sia possibile produrre buchi neri artificiali in laboratorio \Cite{carr}. Anche una stella molto massiccia termina la sua vita come buco nero.
